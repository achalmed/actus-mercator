\documentclass[
  stu,
  floatsintext,
  longtable,
  a4paper,
  nolmodern,
  notxfonts,
  notimes,
  12pt,
  colorlinks=true,linkcolor=blue,citecolor=blue,urlcolor=blue]{apa7}

\usepackage{amsmath}
\usepackage{amssymb}



\usepackage[bidi=default]{babel}
\babelprovide[main,import]{spanish}


% get rid of language-specific shorthands (see #6817):
\let\LanguageShortHands\languageshorthands
\def\languageshorthands#1{}

\RequirePackage{longtable}
\RequirePackage{threeparttablex}

\makeatletter
\renewcommand{\paragraph}{\@startsection{paragraph}{4}{\parindent}%
	{0\baselineskip \@plus 0.2ex \@minus 0.2ex}%
	{-.5em}%
	{\normalfont\normalsize\bfseries\typesectitle}}

\renewcommand{\subparagraph}[1]{\@startsection{subparagraph}{5}{0.5em}%
	{0\baselineskip \@plus 0.2ex \@minus 0.2ex}%
	{-\z@\relax}%
	{\normalfont\normalsize\bfseries\itshape\hspace{\parindent}{#1}\textit{\addperi}}{\relax}}
\makeatother




\usepackage{longtable, booktabs, multirow, multicol, colortbl, hhline, caption, array, float, xpatch}
\usepackage{subcaption}


\renewcommand\thesubfigure{\Alph{subfigure}}
\setcounter{topnumber}{2}
\setcounter{bottomnumber}{2}
\setcounter{totalnumber}{4}
\renewcommand{\topfraction}{0.85}
\renewcommand{\bottomfraction}{0.85}
\renewcommand{\textfraction}{0.15}
\renewcommand{\floatpagefraction}{0.7}

\usepackage{tcolorbox}
\tcbuselibrary{listings,theorems, breakable, skins}
\usepackage{fontawesome5}

\definecolor{quarto-callout-color}{HTML}{909090}
\definecolor{quarto-callout-note-color}{HTML}{0758E5}
\definecolor{quarto-callout-important-color}{HTML}{CC1914}
\definecolor{quarto-callout-warning-color}{HTML}{EB9113}
\definecolor{quarto-callout-tip-color}{HTML}{00A047}
\definecolor{quarto-callout-caution-color}{HTML}{FC5300}
\definecolor{quarto-callout-color-frame}{HTML}{ACACAC}
\definecolor{quarto-callout-note-color-frame}{HTML}{4582EC}
\definecolor{quarto-callout-important-color-frame}{HTML}{D9534F}
\definecolor{quarto-callout-warning-color-frame}{HTML}{F0AD4E}
\definecolor{quarto-callout-tip-color-frame}{HTML}{02B875}
\definecolor{quarto-callout-caution-color-frame}{HTML}{FD7E14}

%\newlength\Oldarrayrulewidth
%\newlength\Oldtabcolsep


\usepackage{hyperref}




\providecommand{\tightlist}{%
  \setlength{\itemsep}{0pt}\setlength{\parskip}{0pt}}
\usepackage{longtable,booktabs,array}
\usepackage{calc} % for calculating minipage widths
% Correct order of tables after \paragraph or \subparagraph
\usepackage{etoolbox}
\makeatletter
\patchcmd\longtable{\par}{\if@noskipsec\mbox{}\fi\par}{}{}
\makeatother
% Allow footnotes in longtable head/foot
\IfFileExists{footnotehyper.sty}{\usepackage{footnotehyper}}{\usepackage{footnote}}
\makesavenoteenv{longtable}

\usepackage{graphicx}
\makeatletter
\newsavebox\pandoc@box
\newcommand*\pandocbounded[1]{% scales image to fit in text height/width
  \sbox\pandoc@box{#1}%
  \Gscale@div\@tempa{\textheight}{\dimexpr\ht\pandoc@box+\dp\pandoc@box\relax}%
  \Gscale@div\@tempb{\linewidth}{\wd\pandoc@box}%
  \ifdim\@tempb\p@<\@tempa\p@\let\@tempa\@tempb\fi% select the smaller of both
  \ifdim\@tempa\p@<\p@\scalebox{\@tempa}{\usebox\pandoc@box}%
  \else\usebox{\pandoc@box}%
  \fi%
}
% Set default figure placement to htbp
\def\fps@figure{htbp}
\makeatother







\usepackage{newtx}

\defaultfontfeatures{Scale=MatchLowercase}
\defaultfontfeatures[\rmfamily]{Ligatures=TeX,Scale=1}





\title{Sistema de Costeo por Órdenes de Producción en la Empresa DULCE
DELEITE S.R.L}


\shorttitle{Costeo por Órdenes: DULCE DELEITE}


\usepackage{etoolbox}


\course{Costos y Presupuestos (EC-543)}
\professor{Econ. Jesús Huamán Palomino}
\duedate{15/12/2017}

\ccoppy{\textcopyright~2017}






\authorsnames[{1},{2},{2},{2},{2},{2},{2}]{Edison Achalma Mendoza,Semnia
Chocce Aguilar,July Miriam Durand Bendezú,Yuri David Fernández
Núñez,Alejandrina Zarcila Galindo Miranda,Brenda Tehalí Gamboa
Rúa,Marisol Huamán Velasque}







\authorsaffiliations{
{Escuela Profesional de Economía, Universidad Nacional de San Cristóbal
de Huamanga},{Escuela Profesional de Economía, Universidad Nacional de
San Cristóbal de Huamanga}}




\leftheader{Mendoza, Aguilar, Bendezú, Núñez, Miranda, Rúa and Velasque}

\date{2017-10-23}


\abstract{El sistema de costeo por órdenes constituye una herramienta
esencial para empresas que operan bajo esquemas de producción
personalizada o por lotes. Este estudio analiza la implementación del
sistema de costeo por órdenes en DULCE DELEITE S.R.L, empresa ayacuchana
dedicada a la elaboración de productos de pastelería. Se documenta el
proceso completo de determinación de costos para una orden de 2,000
unidades de pasteles, identificando y cuantificando los tres elementos
fundamentales del costo de producción: materiales directos (S/
16,730.00), mano de obra directa (S/ 8,192.63) y gastos indirectos de
fabricación (S/ 820.00). Los resultados muestran un costo total de
producción de S/ 25,742.63, equivalente a S/ 12.87 por unidad, con un
margen de utilidad bruta de 35.65\% sobre el precio de venta de S/
20.00. Los materiales directos representan el 65\% del costo total, la
mano de obra directa el 32\% y los gastos indirectos el 3\%. El análisis
revela que cinco insumos principales (chocolate, harina, huevos, leche y
margarina) concentran el 95.64\% del costo de materiales, lo que implica
alta sensibilidad a variaciones de precios en estos insumos críticos. El
estudio demuestra la viabilidad del sistema de costeo por órdenes en
pequeñas y medianas empresas del sector alimentario, proporcionando
información relevante para la fijación de precios competitivos y la
optimización de la rentabilidad empresarial en contextos de producción
por pedidos. }

\keywords{costeo por órdenes, costos de producción, contabilidad de
costos, industria pastelera, gestión empresarial}

\authornote{\par{\addORCIDlink{Edison Achalma
Mendoza}{0000-0001-6996-3364}} 

\par{    Este trabajo fue desarrollado en el marco del curso de Costos y
Presupuestos de la Escuela Profesional de Economía, Universidad Nacional
de San Cristóbal de Huamanga. Agradecemos al Econ. Jesús Huamán
Palomino, docente del curso de Costos y Presupuestos, por su orientación
en el desarrollo de este trabajo. Asimismo, agradecemos a la empresa
DULCE DELEITE S.R.L por permitirnos acceder a información relevante para
el análisis de costos.  Los roles de autor se clasificaron utilizando la
taxonomía de roles de colaborador (CRediT; https://credit.niso.org/) de
la siguiente manera:  Edison Achalma
Mendoza:   conceptualization, writing, methodology, formal
analysis; Semnia Chocce Aguilar:   data curation, investigation; July
Miriam Durand Bendezú:   data curation, investigation; Yuri David
Fernández Núñez:   formal analysis, validation; Alejandrina Zarcila
Galindo Miranda:   investigation, resources; Brenda Tehalí Gamboa
Rúa:   investigation, validation; Marisol Huamán
Velasque:   writing, review \& editing}
\par{La correspondencia relativa a este artículo debe dirigirse a Edison
Achalma Mendoza, Escuela Profesional de Economía, Universidad Nacional
de San Cristóbal de Huamanga, Portal Independencia N°
57, Ayacucho, Ayacucho 5001, Email: \href{mailto:elmer.achalma.09@unsch.edu.pe}{elmer.achalma.09@unsch.edu.pe}}
}

\makeatletter
\let\endoldlt\endlongtable
\def\endlongtable{
\hline
\endoldlt
}
\makeatother

\urlstyle{same}



\makeatletter
\@ifpackageloaded{caption}{}{\usepackage{caption}}
\AtBeginDocument{%
\ifdefined\contentsname
  \renewcommand*\contentsname{Tabla de contenidos}
\else
  \newcommand\contentsname{Tabla de contenidos}
\fi
\ifdefined\listfigurename
  \renewcommand*\listfigurename{Lista de Figuras}
\else
  \newcommand\listfigurename{Lista de Figuras}
\fi
\ifdefined\listtablename
  \renewcommand*\listtablename{Lista de Tablas}
\else
  \newcommand\listtablename{Lista de Tablas}
\fi
\ifdefined\figurename
  \renewcommand*\figurename{Figura}
\else
  \newcommand\figurename{Figura}
\fi
\ifdefined\tablename
  \renewcommand*\tablename{Tabla}
\else
  \newcommand\tablename{Tabla}
\fi
}
\@ifpackageloaded{float}{}{\usepackage{float}}
\floatstyle{ruled}
\@ifundefined{c@chapter}{\newfloat{codelisting}{h}{lop}}{\newfloat{codelisting}{h}{lop}[chapter]}
\floatname{codelisting}{Listado}
\newcommand*\listoflistings{\listof{codelisting}{Lista de Listados}}
\makeatother
\makeatletter
\makeatother
\makeatletter
\@ifpackageloaded{caption}{}{\usepackage{caption}}
\@ifpackageloaded{subcaption}{}{\usepackage{subcaption}}
\makeatother
\makeatletter
\@ifpackageloaded{fontawesome5}{}{\usepackage{fontawesome5}}
\makeatother

% From https://tex.stackexchange.com/a/645996/211326
%%% apa7 doesn't want to add appendix section titles in the toc
%%% let's make it do it
\makeatletter
\xpatchcmd{\appendix}
  {\par}
  {\addcontentsline{toc}{section}{\@currentlabelname}\par}
  {}{}
\makeatother

%% Disable longtable counter
%% https://tex.stackexchange.com/a/248395/211326

\usepackage{etoolbox}

\makeatletter
\patchcmd{\LT@caption}
  {\bgroup}
  {\bgroup\global\LTpatch@captiontrue}
  {}{}
\patchcmd{\longtable}
  {\par}
  {\par\global\LTpatch@captionfalse}
  {}{}
\apptocmd{\endlongtable}
  {\ifLTpatch@caption\else\addtocounter{table}{-1}\fi}
  {}{}
\newif\ifLTpatch@caption
\makeatother

\begin{document}

\maketitle


\hypertarget{toc}{}
\tableofcontents
\newpage
\section[Introduction]{Sistema de Costeo por Órdenes de Producción en la
Empresa DULCE DELEITE S.R.L}

\setcounter{secnumdepth}{3}

\setlength\LTleft{0pt}




El sistema de costos por órdenes de producción constituye una
herramienta fundamental en la contabilidad de gestión empresarial,
especialmente para organizaciones que operan bajo esquemas de producción
personalizada, por lotes o por pedidos específicos. A diferencia de los
sistemas de costeo por procesos, apropiados para producciones masivas y
continuas, el costeo por órdenes permite una asignación precisa y
diferenciada de costos a cada unidad de producción, lote o pedido
individual, facilitando así una mejor toma de decisiones en materia de
fijación de precios, control de costos y análisis de rentabilidad.

En el contexto de la economía peruana, las pequeñas y medianas empresas
(PYMES) del sector alimentario, específicamente del rubro de pastelería,
representan un segmento empresarial significativo que contribuye al
empleo y a la actividad económica local. Sin embargo, muchas de estas
empresas carecen de sistemas formales de contabilidad de costos, lo que
dificulta su capacidad para competir eficientemente en el mercado,
determinar precios de venta adecuados y evaluar su verdadera
rentabilidad. Esta situación se agrava en regiones como Ayacucho, donde
el desarrollo empresarial enfrenta desafíos particulares relacionados
con la formalización, el acceso a tecnología y la capacitación
gerencial.

La monografía surge de la necesidad de documentar y analizar la
aplicación práctica del sistema de costeo por órdenes de producción en
una empresa real del sector pastelero ayacuchano. El estudio se centra
en la empresa DULCE DELEITE S.R.L, entidad dedicada a la elaboración de
productos de pastelería, específicamente en la producción de pasteles
(kekes). La investigación documenta el proceso completo de determinación
de costos para una orden específica de producción de 2,000 unidades de
pasteles, identificando y cuantificando cada uno de los tres elementos
fundamentales del costo de producción: materiales directos, mano de obra
directa y gastos indirectos de fabricación.

El análisis de costeo se basa en información real proporcionada por la
empresa, lo que confiere a este trabajo un valor práctico significativo
tanto para fines académicos como para la gestión empresarial concreta.
Esta característica distingue al estudio de muchos trabajos académicos
que se limitan a presentar casos hipotéticos o ejemplos didácticos
simplificados.

La problemática central que aborda esta monografía puede formularse
mediante la siguiente pregunta de investigación: ¿Cómo se estructura y
cuantifica el sistema de costeo por órdenes de producción en la empresa
DULCE DELEITE S.R.L para la fabricación de pasteles, y qué implicaciones
tiene esta estructura de costos para la determinación del precio de
venta y la rentabilidad empresarial?

Esta pregunta se desglosa en interrogantes específicas: ¿Cuál es la
composición y peso relativo de cada elemento del costo (materiales
directos, mano de obra directa y gastos indirectos) en el costo total de
producción? ¿Qué metodología específica utiliza la empresa para asignar
y distribuir los costos indirectos de fabricación? ¿Cómo se determina el
precio de venta en función de la estructura de costos identificada? ¿Qué
márgenes de utilidad se obtienen con la estructura de costos y precios
actual?

El objetivo principal de esta monografía es \emph{analizar y documentar
el sistema de costeo por órdenes de producción implementado en la
empresa DULCE DELEITE S.R.L para la fabricación de pasteles,
identificando la estructura de costos, los métodos de asignación
empleados y las implicaciones para la gestión empresarial}.

Los objetivos específicos que guían este estudio son:

\begin{enumerate}
\def\labelenumi{\arabic{enumi}.}
\item
  Identificar y cuantificar los costos de materiales directos requeridos
  para la producción de 2,000 unidades de pasteles, especificando
  consumos unitarios y costos por insumo.
\item
  Determinar el costo de la mano de obra directa empleada en el proceso
  productivo, incluyendo las cargas laborales y sociales
  correspondientes según la legislación peruana vigente.
\item
  Calcular los gastos indirectos de fabricación asociados a la orden de
  producción, identificando los rubros que los componen.
\item
  Establecer el costo total de producción y el costo unitario por
  pastel, integrando los tres elementos del costo.
\item
  Analizar la estructura de costos resultante, identificando el peso
  relativo de cada elemento en el costo total.
\item
  Evaluar la determinación del precio de venta en función de la
  estructura de costos y el margen de utilidad esperado.
\end{enumerate}

La realización de esta monografía se justifica por múltiples razones
académicas, prácticas y sociales.

Desde el punto de vista académico, este trabajo contribuye a la
comprensión aplicada de la contabilidad de costos, trasladando los
conceptos teóricos del sistema de costeo por órdenes a un caso concreto
del sector empresarial peruano. Esto permite a los estudiantes y
profesionales de economía y contabilidad observar cómo los principios
teóricos se materializan en la práctica empresarial real, facilitando el
aprendizaje significativo y la transferencia de conocimientos.

En el ámbito empresarial y práctico, la documentación sistemática del
proceso de costeo en DULCE DELEITE S.R.L genera información valiosa para
la propia empresa, proporcionándole herramientas para la toma de
decisiones informadas en áreas como fijación de precios, control de
costos, evaluación de rentabilidad y planificación de la producción.
Asimismo, el estudio puede servir como modelo o referencia para otras
empresas del sector pastelero o de la industria alimentaria en general
que deseen implementar o mejorar sus sistemas de costeo.

Desde una perspectiva social y regional, este trabajo aporta al
fortalecimiento de las capacidades empresariales en Ayacucho, región que
ha experimentado un importante crecimiento económico en las últimas
décadas pero que aún enfrenta brechas significativas en términos de
productividad y competitividad empresarial. La profesionalización de la
gestión de costos en las PYMES ayacuchanas puede contribuir a su
sostenibilidad económica, generación de empleo de calidad y desarrollo
local.

La metodología empleada en esta monografía corresponde a un análisis de
experiencia práctica basado en el estudio de caso de la empresa DULCE
DELEITE S.R.L. El enfoque metodológico se caracteriza por los siguientes
elementos:

\textbf{Tipo de estudio:} Estudio de caso descriptivo-analítico con
enfoque cuantitativo, orientado a documentar y examinar el proceso de
costeo en una empresa específica.

\textbf{Fuentes de información:} Información primaria proporcionada
directamente por la empresa DULCE DELEITE S.R.L, que incluye datos sobre
consumos de materiales, costos de insumos, nómina de trabajadores,
gastos operativos y volúmenes de producción correspondientes al año
2017.

\textbf{Procedimiento de análisis:} El análisis se estructura en cinco
fases: (1) recopilación y organización de datos de costos, (2)
clasificación de los costos en materiales directos, mano de obra directa
y gastos indirectos de fabricación, (3) cálculo de costos unitarios para
cada elemento, (4) determinación del costo total de producción y costo
unitario, y (5) análisis de la estructura de costos y su relación con el
precio de venta.

\textbf{Marco conceptual:} El análisis se fundamenta en los principios
de la contabilidad de costos y específicamente en la metodología de
costeo por órdenes de producción, tal como es presentada en la
literatura especializada.

El documento se organiza en las siguientes secciones principales:

Primero, el Marco Teórico y Conceptual revisa los fundamentos de los
sistemas de costeo, con énfasis en el costeo por órdenes de producción.
Se definen los conceptos clave, se describen los elementos del costo de
producción y se explica la metodología específica del costeo por
órdenes, distinguiéndola de otros sistemas de costeo.

Segundo, la sección de Descripción de la Empresa y del Proceso
Productivo caracteriza a DULCE DELEITE S.R.L en términos de su actividad
económica, estructura organizacional y proceso de elaboración de
pasteles. Esta contextualización es fundamental para comprender cómo se
aplica el sistema de costeo en el caso específico analizado.

Tercero, la sección de Análisis de Costos de Producción constituye el
núcleo del trabajo, presentando el cálculo detallado de cada elemento
del costo: materiales directos (con especificación de insumos, unidades
de medida, consumos y precios), mano de obra directa (incluyendo nómina
de trabajadores, salarios, cargas sociales y prestaciones laborales) y
gastos indirectos de fabricación (identificando rubros como arriendo,
servicios públicos, depreciación, entre otros). Esta sección culmina con
la determinación del costo total de producción y el costo unitario.

Cuarto, en la Discusión de Resultados, se interpreta la estructura de
costos obtenida, se analiza el peso relativo de cada elemento en el
costo total, se evalúa la racionalidad de la determinación del precio de
venta y se comparan estos hallazgos con parámetros de la industria y con
la literatura especializada.

Finalmente, las Conclusiones y Recomendaciones sintetizan los
principales hallazgos del estudio, responden a la pregunta de
investigación planteada y ofrecen sugerencias tanto para la empresa
analizada como para otras organizaciones del sector que deseen
implementar o mejorar sus sistemas de costeo.

\section{Marco Teórico y
Conceptual}\label{marco-teuxf3rico-y-conceptual}

El análisis de costos de producción en empresas manufactureras requiere
de un sólido fundamento conceptual que permita comprender los
principios, metodologías y herramientas empleadas en la contabilidad de
gestión. Esta sección presenta los conceptos fundamentales del costeo
por órdenes de producción, los elementos del costo, las diferencias con
otros sistemas de costeo y el marco teórico específico aplicable al caso
de estudio.

\subsection{Fundamentos de la Contabilidad de
Costos}\label{fundamentos-de-la-contabilidad-de-costos}

\subsubsection{Concepto y Objetivos de la Contabilidad de
Costos}\label{concepto-y-objetivos-de-la-contabilidad-de-costos}

La contabilidad de costos es una rama especializada de la contabilidad
que se dedica a la identificación, medición, acumulación, análisis,
preparación, interpretación y comunicación de información sobre los
costos incurridos por una organización en sus procesos productivos y
operativos. A diferencia de la contabilidad financiera, que se orienta
principalmente a usuarios externos y está regulada por normas contables
estandarizadas, la contabilidad de costos se enfoca en proporcionar
información relevante para la toma de decisiones internas de la gestión
empresarial.

Los objetivos principales de la contabilidad de costos incluyen: (1) la
determinación del costo de los productos fabricados o servicios
prestados, (2) el control de los costos mediante la comparación entre
costos reales y estándares o presupuestados, (3) el suministro de
información para la planeación de utilidades y la toma de decisiones de
inversión, (4) la contribución a la fijación de precios de venta, y (5)
la valuación de inventarios para efectos de estados financieros.

En el contexto empresarial contemporáneo, la contabilidad de costos ha
evolucionado más allá de su función tradicional de valuación de
inventarios, convirtiéndose en una herramienta estratégica que apoya la
competitividad empresarial mediante la generación de información para la
gestión del valor, el análisis de rentabilidad de productos y clientes,
y la optimización de procesos.

\subsubsection{Clasificación de los
Costos}\label{clasificaciuxf3n-de-los-costos}

La comprensión adecuada de los sistemas de costeo requiere el dominio de
las diversas clasificaciones de costos. Los costos pueden clasificarse
según múltiples criterios, siendo los más relevantes para este estudio
los siguientes:

\textbf{Por su identificación con el producto:}

\begin{itemize}
\item
  \emph{Costos directos}: aquellos que pueden identificarse y asignarse
  específicamente a un producto, departamento o actividad sin necesidad
  de prorrateo. Ejemplos típicos incluyen las materias primas utilizadas
  en la fabricación de un producto específico y la mano de obra de
  trabajadores que se dedican exclusivamente a la elaboración de dicho
  producto.
\item
  \emph{Costos indirectos}: aquellos que no pueden identificarse
  directamente con un producto específico, sino que benefician a varios
  productos o al proceso productivo en general. Su asignación a
  productos individuales requiere el uso de bases de distribución o
  prorrateo. Ejemplos incluyen el arriendo de la planta de producción,
  la supervisión general, los servicios públicos y la depreciación de
  maquinaria de uso múltiple.
\end{itemize}

\textbf{Por su comportamiento frente a cambios en el volumen de
producción:}

\begin{itemize}
\item
  \emph{Costos variables}: aquellos que varían en proporción directa con
  los cambios en el nivel de actividad o volumen de producción. Un
  ejemplo claro son las materias primas: a mayor producción, mayor
  consumo de materiales.
\item
  \emph{Costos fijos}: aquellos que permanecen constantes en su monto
  total dentro de un rango relevante de actividad, independientemente de
  las variaciones en el volumen de producción. Ejemplos típicos son el
  arriendo de instalaciones, los salarios de personal administrativo
  permanente y las primas de seguros anuales.
\item
  \emph{Costos mixtos o semivariables}: aquellos que contienen tanto un
  componente fijo como uno variable. Por ejemplo, el costo de energía
  eléctrica puede tener una tarifa fija mensual más un cargo variable
  por kilowatt-hora consumido.
\end{itemize}

\textbf{Por su función en la empresa:}

\begin{itemize}
\item
  \emph{Costos de producción o fabricación}: relacionados con la
  elaboración de productos. Se subdividen en materiales directos, mano
  de obra directa y gastos indirectos de fabricación.
\item
  \emph{Costos de distribución o venta}: relacionados con la
  comercialización y entrega de productos a los clientes (publicidad,
  comisiones de ventas, transporte, etc.).
\item
  \emph{Costos de administración}: relacionados con la dirección y
  gestión general de la empresa (salarios de gerentes, gastos de
  oficina, honorarios profesionales, etc.).
\end{itemize}

\textbf{Por el momento en que se registran:}

\begin{itemize}
\item
  \emph{Costos históricos o reales}: aquellos que efectivamente se
  incurrieron y se registran después de realizados.
\item
  \emph{Costos predeterminados}: aquellos que se calculan antes de
  iniciar la producción, basándose en estimaciones o estándares. Se
  subdividen en costos estimados y costos estándar.
\end{itemize}

Esta taxonomía de costos es fundamental para la implementación de
cualquier sistema de costeo, ya que determina cómo se acumulan, asignan
y reportan los diferentes tipos de costos en el proceso de determinación
del costo de producción.

\subsection{Elementos del Costo de
Producción}\label{elementos-del-costo-de-producciuxf3n}

El costo de producción de un bien manufacturado se compone de tres
elementos fundamentales, conocidos en la literatura anglosajona como
\emph{prime cost} (costo primo, suma de materiales directos y mano de
obra directa) y \emph{conversion cost} (costo de conversión, suma de
mano de obra directa y gastos indirectos de fabricación). Estos tres
elementos son:

\subsubsection{Materiales Directos}\label{materiales-directos}

Los materiales directos, también denominados materia prima directa,
comprenden todos aquellos materiales que se convierten en parte integral
del producto terminado y cuyos costos pueden identificarse directamente
con el producto de manera económicamente factible. En el caso de una
empresa pastelera como DULCE DELEITE S.R.L, los materiales directos
incluirían ingredientes como harina, huevos, azúcar, margarina, leche,
chocolate, entre otros insumos que se incorporan físicamente al pastel.

Para que un material se considere directo, debe cumplir dos condiciones:
(1) formar parte física del producto terminado, y (2) ser susceptible de
medición directa en relación con el producto, de modo que el costo de
rastreo sea menor que el beneficio de tener información precisa. Algunos
materiales, aunque físicamente se incorporan al producto, se consideran
indirectos por razones de inmaterialidad o impracticabilidad de medición
(por ejemplo, el hilo utilizado en la confección de una prenda de
vestir, cuyo costo unitario es tan pequeño que no justifica su
seguimiento individual).

El costo de los materiales directos incluye no solo el precio de compra
neto de descuentos comerciales, sino también todos los costos necesarios
para poner el material en condiciones de uso: fletes de compra, seguros
de transporte, derechos de importación (si aplica), costos de inspección
y recepción, entre otros. Sin embargo, los descuentos por pronto pago
generalmente se tratan como ingresos financieros y no reducen el costo
del material.

\subsubsection{Mano de Obra Directa}\label{mano-de-obra-directa}

La mano de obra directa representa el esfuerzo humano que puede
identificarse directamente con la elaboración de un producto específico.
Comprende el costo del trabajo de los empleados que manipulan
directamente los materiales para convertirlos en producto terminado o
que operan maquinaria de producción dedicada específicamente a un
producto.

El costo de la mano de obra directa incluye no solo el salario base de
los trabajadores, sino también todas las cargas sociales y prestaciones
laborales establecidas por la legislación vigente. En el caso peruano,
estas cargas incluyen:

\begin{itemize}
\tightlist
\item
  Aportaciones del empleador al seguro social de salud (EsSalud): 9\% de
  la remuneración.
\item
  Seguro Complementario de Trabajo de Riesgo (SCTR): variable según el
  nivel de riesgo de la actividad, generalmente entre 0.51\% y 1.56\%.
\item
  Aportaciones al Servicio Nacional de Adiestramiento en Trabajo
  Industrial (SENATI): 0.75\% de la remuneración para empresas
  industriales con más de 20 trabajadores.
\item
  Compensación por Tiempo de Servicios (CTS): equivalente a
  aproximadamente 8.33\% mensual de la remuneración (una remuneración
  mensual dividida en 12 meses).
\item
  Gratificaciones legales (Fiestas Patrias y Navidad): equivalente a
  aproximadamente 16.67\% mensual (dos remuneraciones al año divididas
  en 12 meses).
\item
  Vacaciones: equivalente a aproximadamente 8.33\% mensual.
\end{itemize}

Adicionalmente, los trabajadores aportan el 13\% de su remuneración para
el sistema de pensiones (ONP o AFP), pero este monto no constituye un
costo adicional para el empleador, sino una retención que se descuenta
del salario del trabajador.

La distinción entre mano de obra directa e indirecta es crucial en el
costeo por órdenes. Los salarios de trabajadores cuyo esfuerzo puede
rastrearse a órdenes específicas (como operarios de línea de producción)
se consideran costos directos, mientras que los salarios de personal de
supervisión, mantenimiento o apoyo general se clasifican como costos
indirectos de fabricación.

\subsubsection{Gastos Indirectos de
Fabricación}\label{gastos-indirectos-de-fabricaciuxf3n}

Los gastos indirectos de fabricación (GIF), también denominados costos
indirectos de manufactura o \emph{manufacturing overhead}, comprenden
todos los costos de producción que no pueden identificarse directamente
con unidades específicas de producto. Esta categoría incluye tres
subgrupos:

\textbf{Materiales indirectos:} materiales necesarios para el proceso de
producción pero que no se convierten en parte identificable del producto
terminado, o cuyo costo de rastreo no justifica el esfuerzo. Ejemplos:
lubricantes para maquinaria, combustibles, materiales de limpieza de
planta, herramientas menores.

\textbf{Mano de obra indirecta:} salarios y cargas sociales del personal
de producción que no trabaja directamente en la elaboración de productos
específicos. Ejemplos: supervisores de planta, personal de
mantenimiento, inspectores de calidad, personal de almacén de materiales
y productos terminados.

\textbf{Otros gastos indirectos de fabricación:} todos los demás costos
de producción no clasificables como materiales directos, mano de obra
directa, materiales indirectos o mano de obra indirecta. Ejemplos:
arriendo de planta, servicios públicos (electricidad, agua, gas),
depreciación de maquinaria y edificios de producción, seguros de planta,
impuestos sobre propiedad industrial.

Una característica fundamental de los GIF es que su asignación a
productos específicos requiere el uso de bases de distribución o
prorrateo, ya que por definición no pueden rastrearse directamente a las
unidades producidas. Las bases de distribución más comunes incluyen:
horas de mano de obra directa, horas-máquina, costo de materiales
directos, costo de mano de obra directa o unidades producidas. La
elección de la base apropiada debe reflejar la relación causal entre el
costo indirecto y el factor de actividad seleccionado.

\subsection{Sistemas de Costeo: Clasificación
General}\label{sistemas-de-costeo-clasificaciuxf3n-general}

Existen múltiples criterios para clasificar los sistemas de costeo. Los
dos criterios más relevantes para este estudio son: (1) según el método
de acumulación de costos, y (2) según el tratamiento de los costos fijos
de producción.

\subsubsection{Según el Método de Acumulación de
Costos}\label{seguxfan-el-muxe9todo-de-acumulaciuxf3n-de-costos}

\textbf{Sistema de Costeo por Órdenes de Producción}

El sistema de costeo por órdenes de producción (también llamado costeo
por órdenes específicas, por lotes o por pedidos) es apropiado para
empresas que fabrican productos o prestan servicios según
especificaciones de clientes individuales, o que producen en lotes
identificables de productos. Cada orden, lote o pedido se trata como una
unidad de costeo separada, acumulándose los costos de materiales
directos, mano de obra directa y gastos indirectos de fabricación para
cada orden específica.

Este sistema se caracteriza por:

\begin{itemize}
\tightlist
\item
  La producción se realiza en función de órdenes específicas de clientes
  o lotes predeterminados.
\item
  Cada orden de producción se identifica mediante un número o código
  único.
\item
  Los costos se acumulan por orden de producción mediante una hoja de
  costos por orden.
\item
  El costo unitario se calcula dividiendo el costo total de la orden
  entre el número de unidades producidas en esa orden.
\item
  Es posible que diferentes órdenes tengan costos unitarios diferentes,
  incluso para productos similares, debido a variaciones en
  especificaciones, volumen o eficiencia de producción.
\end{itemize}

El sistema de costeo por órdenes es el más adecuado para industrias
como: construcción, imprenta, mueblería personalizada, reparaciones,
consultoría profesional, y en general cualquier actividad donde cada
trabajo, proyecto o pedido sea diferente de los demás.

\textbf{Sistema de Costeo por Procesos}

El sistema de costeo por procesos es apropiado para empresas que
fabrican productos homogéneos de manera continua y masiva, donde las
unidades individuales no pueden distinguirse entre sí durante el proceso
de producción. Los costos se acumulan por proceso o departamento durante
un período (generalmente un mes), y el costo unitario se determina
dividiendo el costo total del proceso entre el número de unidades
equivalentes producidas en ese período.

Este sistema se caracteriza por:

\begin{itemize}
\tightlist
\item
  Producción continua y homogénea de productos estandarizados.
\item
  Los costos se acumulan por departamento o centro de costos durante un
  período definido.
\item
  El flujo de unidades es continuo a través de varios procesos o
  departamentos.
\item
  Se utiliza el concepto de unidades equivalentes para manejar la
  producción en proceso al final del período.
\item
  El costo unitario es un promedio del período, aplicable a todas las
  unidades producidas.
\end{itemize}

El costeo por procesos es típico de industrias como: química, textil,
alimentaria (procesamiento masivo), papel, cemento, refinación de
petróleo.

\textbf{Diferencias clave entre costeo por órdenes y costeo por
procesos}

La Tabla~\ref{tbl-comparacion-sistemas} presenta las diferencias
fundamentales entre ambos sistemas:

\begin{table}

{\caption{{Comparación entre Sistema de Costeo por Órdenes y por
Procesos}{\label{tbl-comparacion-sistemas}}}}

\begin{longtable}[]{@{}
  >{\raggedright\arraybackslash}p{(\linewidth - 4\tabcolsep) * \real{0.2807}}
  >{\raggedright\arraybackslash}p{(\linewidth - 4\tabcolsep) * \real{0.3509}}
  >{\raggedright\arraybackslash}p{(\linewidth - 4\tabcolsep) * \real{0.3684}}@{}}
\toprule\noalign{}
\begin{minipage}[b]{\linewidth}\raggedright
Característica
\end{minipage} & \begin{minipage}[b]{\linewidth}\raggedright
Costeo por Órdenes
\end{minipage} & \begin{minipage}[b]{\linewidth}\raggedright
Costeo por Procesos
\end{minipage} \\
\midrule\noalign{}
\endhead
\bottomrule\noalign{}
\endlastfoot
Tipo de producción & Por pedidos, lotes o proyectos específicos &
Continua y masiva \\
Homogeneidad del producto & Productos diferentes entre órdenes &
Productos homogéneos \\
Unidad de costeo & Orden específica & Departamento/proceso por
período \\
Cálculo del costo unitario & Costo total de orden / unidades de orden &
Costo total de proceso / unidades equivalentes \\
Flexibilidad & Alta (cada orden diferente) & Baja (producción
estandarizada) \\
Momento de cálculo & Al terminar cada orden & Al finalizar el período
contable \\
Ejemplos de industrias & Construcción, imprenta, muebles & Química,
alimentaria, textil \\
\end{longtable}

\noindent \emph{Nota.} Elaboración~propia basada en literatura
especializada en contabilidad de costos.

\end{table}

\subsubsection{Según el Tratamiento de los Costos Fijos de
Producción}\label{seguxfan-el-tratamiento-de-los-costos-fijos-de-producciuxf3n}

\textbf{Costeo por Absorción (Costeo Total)}

En el costeo por absorción, todos los costos de producción, tanto
variables como fijos, se consideran costos del producto y se incluyen en
el costo de inventarios. Bajo este método, los GIF fijos (como el
arriendo de planta o la depreciación) se prorratean a las unidades
producidas y forman parte del costo unitario del producto.

Este método es el requerido por las Normas Internacionales de
Información Financiera (NIIF) y por las normas tributarias de la mayoría
de países, incluido el Perú, para la valuación de inventarios en los
estados financieros de propósito general.

\textbf{Costeo Variable (Costeo Directo)}

En el costeo variable, solo los costos variables de producción se
consideran costos del producto, mientras que los costos fijos de
producción se tratan como costos del período y se cargan directamente a
resultados. Este método es útil para el análisis de decisiones de corto
plazo y el análisis de contribución marginal, pero no es aceptable para
reportes financieros externos bajo NIIF.

Para los propósitos de este estudio, que analiza la estructura de costos
de una empresa específica en un contexto de gestión interna, se utiliza
el enfoque de costeo por absorción, aunque sin una distinción explícita
entre costos fijos y variables en la asignación de GIF, dado que el
sistema implementado en DULCE DELEITE S.R.L no hace esta diferenciación.

\subsection{El Sistema de Costeo por Órdenes de Producción: Aspectos
Específicos}\label{el-sistema-de-costeo-por-uxf3rdenes-de-producciuxf3n-aspectos-especuxedficos}

\subsubsection{Documentos Básicos del
Sistema}\label{documentos-buxe1sicos-del-sistema}

La implementación del costeo por órdenes requiere de documentos
específicos para la acumulación y control de costos:

\textbf{Orden de Producción}

Es el documento que autoriza la fabricación de una cantidad específica
de un producto o lote. Contiene información como: número de orden,
descripción del producto, cantidad a fabricar, fecha de inicio, fecha de
entrega requerida, especificaciones técnicas, departamentos
involucrados. Este documento inicia el proceso de acumulación de costos
para la orden.

\textbf{Hoja de Costos por Orden}

Es el documento fundamental del sistema, donde se acumulan todos los
costos incurridos en la fabricación de una orden específica. Se
estructura en tres secciones principales: (1) materiales directos, (2)
mano de obra directa, y (3) gastos indirectos de fabricación. Cada
sección registra los costos conforme se incurren durante la ejecución de
la orden.

La hoja de costos por orden sirve como mayor auxiliar que sustenta los
saldos de la cuenta de inventario de productos en proceso. Al finalizar
la orden, el costo total acumulado en la hoja se transfiere a inventario
de productos terminados.

\textbf{Requisiciones de Materiales}

Documento que autoriza la entrega de materiales del almacén al área de
producción. Especifica: número de orden de producción, tipo de material,
cantidad solicitada, fecha, firma del solicitante y del encargado de
almacén. Las requisiciones de materiales directos se registran
directamente en la hoja de costos de la orden correspondiente.

\textbf{Tarjetas de Tiempo o Reportes de Horas}

Documentos que registran el tiempo que los trabajadores dedican a cada
orden de producción. Permiten asignar los costos de mano de obra directa
a las órdenes específicas. En sistemas más sofisticados, se utilizan
sistemas electrónicos de seguimiento de tiempo (tarjetas electrónicas,
software de gestión de producción).

\subsubsection{Proceso de Acumulación de
Costos}\label{proceso-de-acumulaciuxf3n-de-costos}

El proceso de costeo por órdenes sigue una secuencia lógica:

\begin{enumerate}
\def\labelenumi{\arabic{enumi}.}
\item
  \textbf{Emisión de la Orden de Producción}: el departamento de ventas
  o planificación emite una orden de producción especificando qué se va
  a fabricar, en qué cantidad y para cuándo.
\item
  \textbf{Apertura de la Hoja de Costos}: se crea una hoja de costos
  para la orden, asignándole un número único de identificación.
\item
  \textbf{Acumulación de Materiales Directos}: conforme se retiran
  materiales del almacén mediante requisiciones, se registran en la hoja
  de costos de la orden correspondiente, valorándose generalmente al
  costo promedio, PEPS (primeras entradas, primeras salidas) o algún
  otro método de valuación de inventarios.
\item
  \textbf{Acumulación de Mano de Obra Directa}: con base en las tarjetas
  de tiempo o reportes de horas, se asigna a cada orden el costo de la
  mano de obra directa utilizada, incluyendo salarios base y cargas
  sociales.
\item
  \textbf{Aplicación de Gastos Indirectos de Fabricación}: dado que los
  GIF no pueden rastrearse directamente a las órdenes, se aplican
  mediante una tasa predeterminada de GIF. Esta tasa se calcula al
  inicio del período dividiendo los GIF presupuestados entre el nivel
  esperado de la base de asignación (ej. horas de mano de obra directa
  presupuestadas). Durante el período, los GIF se aplican a cada orden
  multiplicando la tasa predeterminada por el nivel real de la base de
  asignación de esa orden.
\item
  \textbf{Cierre de la Orden}: cuando la orden se completa, se suman los
  tres elementos del costo en la hoja de costos, obteniéndose el costo
  total de la orden. Este costo se divide entre el número de unidades
  producidas para obtener el costo unitario. La orden se transfiere de
  inventario de productos en proceso a inventario de productos
  terminados.
\end{enumerate}

\subsubsection{Métodos de Asignación de Gastos Indirectos de
Fabricación}\label{muxe9todos-de-asignaciuxf3n-de-gastos-indirectos-de-fabricaciuxf3n}

La asignación de GIF es uno de los aspectos más complejos del costeo por
órdenes, dado que estos costos, por definición, no pueden identificarse
directamente con órdenes específicas. Existen varios enfoques:

\textbf{Tasa Única de Asignación}

Se utiliza una sola tasa de GIF para toda la planta, calculada como:

\[
\text{Tasa de GIF} = \frac{\text{GIF Presupuestados Totales}}{\text{Base de Asignación Presupuestada Total}}
\]

Este método es simple pero puede ser impreciso si diferentes productos o
departamentos consumen recursos indirectos en proporciones muy
diferentes.

\textbf{Tasas Departamentales de Asignación}

Se calculan tasas separadas para cada departamento o centro de costos,
reconociendo que diferentes áreas de producción pueden tener estructuras
de costos indirectos diferentes. Por ejemplo, un departamento intensivo
en maquinaria puede usar horas-máquina como base, mientras que uno
intensivo en trabajo manual puede usar horas de mano de obra directa.

\textbf{Costeo Basado en Actividades (ABC)}

Método más sofisticado que identifica actividades que causan costos
indirectos (ej. preparación de máquinas, inspección de calidad, manejo
de materiales) y asigna los costos de estas actividades a los productos
según el consumo de actividades. Aunque más preciso, es también más
complejo y costoso de implementar.

\subsubsection{Tratamiento de Unidades Defectuosas, Desperdicios y
Material de
Desecho}\label{tratamiento-de-unidades-defectuosas-desperdicios-y-material-de-desecho}

En el proceso productivo es común que se generen unidades que no cumplen
los estándares de calidad, así como desperdicios o material de desecho.
El tratamiento contable de estos elementos afecta el costo del producto:

\textbf{Unidades Defectuosas}

Unidades que no cumplen con los estándares de producción y que deben
reprocesarse para poder venderse. El costo del reproceso puede tratarse
de dos formas:

\begin{itemize}
\tightlist
\item
  \emph{Cargado a la orden específica}: si el defecto es atribuible a
  las especificaciones particulares de esa orden o a errores en su
  producción.
\item
  \emph{Cargado a los GIF}: si el nivel de defectos está dentro de lo
  normal esperado para el proceso productivo en general.
\end{itemize}

\textbf{Unidades Dañadas}

Unidades que no cumplen los estándares y no pueden reprocesarse
económicamente, por lo que se descartan o venden a precio de salvamento.
El costo neto de las unidades dañadas (costo de producción incurrido
menos valor de salvamento) puede:

\begin{itemize}
\tightlist
\item
  Cargarse a la orden específica si el daño es atribuible a esa orden.
\item
  Cargarse a los GIF si es parte del deterioro normal del proceso.
\end{itemize}

\textbf{Desperdicios}

Materia prima que queda del proceso de producción y que no puede
reutilizarse para el mismo propósito, pero puede venderse a terceros a
valor nominal. El valor de salvamento de los desperdicios generalmente
se deduce del costo de materiales directos o se acredita a los GIF,
reduciendo así el costo del producto.

\textbf{Material de Desecho}

Parte de las materias primas que queda después de la producción sin
valor de reventa. Simplemente se descarta, y su costo está
implícitamente incluido en el costo de materiales directos (es decir, se
compra más material del que efectivamente se incorpora al producto, para
cubrir el desecho normal).

En el caso de DULCE DELEITE S.R.L, aunque el sistema de costeo
implementado no hace un tratamiento explícito y separado de estos
conceptos, es importante reconocer que en la producción de pasteles
pueden generarse mermas (por ejemplo, masa que se adhiere a los moldes,
recortes al dar forma a los pasteles) que están implícitamente
consideradas en los consumos de materiales directos especificados en las
fórmulas de producción.

\subsection{Marco Normativo y Regulatorio en el
Perú}\label{marco-normativo-y-regulatorio-en-el-peruxfa}

\subsubsection{Legislación Laboral
Aplicable}\label{legislaciuxf3n-laboral-aplicable}

El cálculo del costo de mano de obra en el Perú debe considerar el marco
normativo establecido por el Ministerio de Trabajo y Promoción del
Empleo (MTPE) y otras entidades reguladoras. Las principales normas
aplicables durante el período analizado (2017) incluían:

\begin{itemize}
\tightlist
\item
  \textbf{Remuneración Mínima Vital}: establecida en S/ 850.00 mensuales
  mediante Decreto Supremo N° 005-2016-TR (vigente hasta marzo de 2018).
\item
  \textbf{Jornada de Trabajo}: máximo 8 horas diarias o 48 horas
  semanales, según Decreto Legislativo N° 854.
\item
  \textbf{Gratificaciones}: dos gratificaciones legales al año (Fiestas
  Patrias y Navidad) equivalentes a una remuneración mensual cada una,
  según Ley N° 27735.
\item
  \textbf{Compensación por Tiempo de Servicios (CTS)}: depósito
  semestral equivalente a una remuneración mensual al año, conforme al
  Texto Único Ordenado del Decreto Legislativo N° 650.
\item
  \textbf{Vacaciones}: 30 días calendario por cada año completo de
  servicios, conforme al Decreto Legislativo N° 713.
\item
  \textbf{EsSalud}: aportación del empleador del 9\% de la remuneración
  al Seguro Social de Salud, según Ley N° 27056.
\item
  \textbf{ONP o AFP}: aportación del trabajador del 13\% (ONP) o
  variable según AFP elegida, que no constituye costo para el empleador
  sino retención del salario del trabajador.
\item
  \textbf{SCTR}: Seguro Complementario de Trabajo de Riesgo, obligatorio
  para actividades de riesgo, con tasas variables según nivel de riesgo,
  conforme a la Ley N° 26790.
\end{itemize}

\subsubsection{Normas Contables y
Tributarias}\label{normas-contables-y-tributarias}

Para efectos de presentación de estados financieros, las empresas
peruanas deben aplicar las Normas Internacionales de Información
Financiera (NIIF), adoptadas en el Perú mediante resoluciones del
Consejo Normativo de Contabilidad. La valuación de inventarios se rige
por la NIC 2 \emph{Inventarios}, que establece que los inventarios deben
medirse al menor entre el costo y el valor neto realizable, y que el
costo de los inventarios de productos manufacturados debe incluir
materiales directos, mano de obra directa y una porción de los GIF fijos
y variables.

Para efectos tributarios, la Ley del Impuesto a la Renta (Texto Único
Ordenado aprobado por Decreto Supremo N° 179-2004-EF) y su reglamento
establecen que el costo de producción se determina conforme a las normas
contables, siempre que estas no contravengan disposiciones tributarias
específicas.

\section{Descripción de la Empresa y del Proceso
Productivo}\label{descripciuxf3n-de-la-empresa-y-del-proceso-productivo}

\subsection{Caracterización de DULCE DELEITE
S.R.L}\label{caracterizaciuxf3n-de-dulce-deleite-s.r.l}

\subsubsection{Razón Social y Actividad
Económica}\label{razuxf3n-social-y-actividad-econuxf3mica}

DULCE DELEITE S.R.L es una empresa peruana constituida como Sociedad de
Responsabilidad Limitada, dedicada a la elaboración y comercialización
de productos de pastelería y repostería. Su actividad económica
principal se clasifica dentro de la División 10 (Elaboración de
productos alimenticios), Grupo 107 (Elaboración de otros productos
alimenticios) de la Clasificación Industrial Internacional Uniforme
(CIIU) Revisión 4, adaptada para el Perú.

La empresa opera en la ciudad de Ayacucho, región que en los últimos
años ha experimentado un crecimiento económico significativo impulsado
por la inversión pública, el comercio y el turismo. El sector de
alimentos preparados y panadería-pastelería ha crecido paralelamente,
atendiendo tanto a la demanda local como a eventos especiales,
celebraciones y el mercado institucional (colegios, empresas,
instituciones públicas).

\subsubsection{Estructura Organizacional y Recursos
Humanos}\label{estructura-organizacional-y-recursos-humanos}

Aunque la información proporcionada no detalla la estructura
organizacional completa de la empresa, el análisis de costos de mano de
obra revela que para la orden de producción analizada se emplearon cinco
trabajadores de producción directa, además de al menos un trabajador
administrativo (cuyo salario aparece en el cálculo de mano de obra
indirecta).

La estructura básica de una empresa de este tipo generalmente incluye:

\begin{itemize}
\tightlist
\item
  \textbf{Área de producción}: operarios que preparan la masa, hornean,
  decoran y empacan los productos.
\item
  \textbf{Área de ventas}: personal encargado de atender pedidos,
  coordinar entregas y gestionar relaciones con clientes.
\item
  \textbf{Área administrativa}: encargada de contabilidad, compras,
  recursos humanos y gestión general.
\end{itemize}

Dado que se trata de una S.R.L, la gestión está a cargo de uno o más
gerentes designados por los socios, quienes toman las decisiones
estratégicas del negocio.

\subsubsection{Infraestructura y
Equipamiento}\label{infraestructura-y-equipamiento}

El análisis de los gastos indirectos de fabricación proporciona indicios
sobre la infraestructura y equipamiento de la empresa:

\begin{itemize}
\tightlist
\item
  \textbf{Instalaciones físicas}: la empresa opera en un local arrendado
  (evidenciado por el rubro ``ARRIENDO'' en los GIF por un monto de S/
  350.00 mensuales).
\item
  \textbf{Servicios básicos}: cuenta con suministro de agua potable,
  energía eléctrica (con consumo significativo de S/ 300.00 mensuales,
  atribuible al uso de hornos y otros equipos), telefonía e internet.
\item
  \textbf{Equipamiento de producción}: aunque no se detalla
  específicamente, una operación de pastelería requiere hornos
  industriales o semi-industriales, batidoras, mesas de trabajo, moldes,
  utensilios diversos y equipos de refrigeración. La depreciación
  mensual de S/ 50.00 registrada en los GIF sugiere la existencia de
  maquinaria y equipo con valor contable sujeto a depreciación.
\end{itemize}

\subsubsection{Mercado y Clientes}\label{mercado-y-clientes}

Si bien el estudio de caso no proporciona información detallada sobre el
mercado y la cartera de clientes, puede inferirse que DULCE DELEITE
S.R.L opera en un esquema de producción por pedidos o por lotes,
característica que justifica precisamente la aplicación del sistema de
costeo por órdenes.

El tamaño de la orden analizada (2,000 unidades de pasteles) sugiere que
la empresa atiende pedidos de volumen medio, posiblemente para eventos
corporativos, institucionales o comerciales (distribución a tiendas
minoristas), además de potencialmente atender pedidos menores de
clientes individuales para eventos familiares.

El precio de venta fijado en S/ 20.00 por unidad la posiciona en un
segmento de mercado de precio moderado a medio, considerando que el
costo de producción es de S/ 12.87 por unidad.

\subsection{Proceso de Elaboración de
Pasteles}\label{proceso-de-elaboraciuxf3n-de-pasteles}

\subsubsection{Descripción General del
Proceso}\label{descripciuxf3n-general-del-proceso}

La elaboración de pasteles (kekes) en una operación de pastelería sigue
un proceso productivo relativamente estandarizado, aunque con posibles
variaciones según el tipo específico de pastel y las preferencias del
fabricante. Con base en la información de materiales directos
proporcionada, puede reconstruirse el proceso seguido por DULCE DELEITE
S.R.L:

\textbf{Etapa 1: Preparación de Ingredientes y Pesado}

Los trabajadores seleccionan y pesan los ingredientes según la fórmula
establecida para el tipo de pastel a elaborar. Los ingredientes
identificados en la orden analizada incluyen:

\begin{itemize}
\tightlist
\item
  Harina (base estructural del pastel)
\item
  Huevos (proporciona estructura, humedad y sabor)
\item
  Azúcar (endulzante y contribuye a textura)
\item
  Margarina sin sal (aporta grasa, humedad y sabor)
\item
  Leche (líquido que activa el gluten y aporta sabor)
\item
  Polvo para hornear (agente leudante que hace que la masa suba)
\item
  Esencia de vainilla (saborizante)
\item
  Chocolate (ingrediente característico, posiblemente para cobertura o
  relleno)
\end{itemize}

La precisión en el pesado es crucial para la calidad y consistencia del
producto final, así como para el control de costos, ya que variaciones
en las cantidades afectan tanto la calidad como el costo unitario.

\textbf{Etapa 2: Mezclado de Ingredientes}

Los ingredientes se mezclan siguiendo una secuencia específica para
lograr la textura y consistencia adecuadas de la masa. El proceso típico
incluye:

\begin{itemize}
\tightlist
\item
  Batido de margarina con azúcar hasta obtener una mezcla cremosa
\item
  Incorporación gradual de huevos
\item
  Alternancia en la adición de ingredientes secos (harina, polvo de
  hornear) e ingredientes líquidos (leche, esencia de vainilla)
\item
  Mezclado hasta obtener una masa homogénea sin grumos
\end{itemize}

El tiempo y método de mezclado son críticos: un mezclado insuficiente
resulta en una masa grumosa e irregular, mientras que un mezclado
excesivo puede desarrollar demasiado el gluten de la harina, resultando
en un producto denso y duro.

\textbf{Etapa 3: Moldeado y Llenado}

La masa preparada se distribuye en moldes previamente engrasados o
forrados. El llenado debe ser uniforme para asegurar que todos los
pasteles tengan tamaño y peso similares (importante para la consistencia
del producto y el control de costos unitarios).

\textbf{Etapa 4: Horneado}

Los moldes con la masa se introducen en hornos precalentados a la
temperatura adecuada (generalmente entre 170°C y 180°C para pasteles).
El tiempo de horneado depende del tamaño de los pasteles y las
características del horno, típicamente entre 25 y 45 minutos para
pasteles de tamaño individual a mediano.

Durante el horneado, el calor hace que el polvo de hornear genere
dióxido de carbono, expandiendo la masa; las proteínas del huevo y el
gluten de la harina se coagulan, dando estructura; el agua se evapora
parcialmente; y la superficie se dora debido a la caramelización de
azúcares y la reacción de Maillard.

El control de temperatura y tiempo es crucial: un horneado insuficiente
resulta en un producto crudo en el centro, mientras que un horneado
excesivo produce un pastel seco y duro.

\textbf{Etapa 5: Enfriamiento}

Una vez completado el horneado, los pasteles se retiran del horno y se
dejan enfriar parcialmente en los moldes antes de desmoldarlos.
Posteriormente, se enfrían completamente en rejillas que permiten la
circulación de aire. El enfriamiento gradual previene el colapso de la
estructura del pastel.

\textbf{Etapa 6: Decorado y Acabado}

Dependiendo del tipo de producto, puede incluirse una etapa de decorado
con chocolate (mencionado en los materiales directos), glaseado, relleno
o coberturas. En el caso analizado, la inclusión de chocolate como
material directo (0.200 kg por unidad, equivalente a 200 gramos) sugiere
que los pasteles llevan algún tipo de cobertura o inclusión de chocolate
significativa.

\textbf{Etapa 7: Empaque}

Los pasteles terminados y enfriados se empacan individualmente o en
grupos, dependiendo del destino final (venta individual, distribución a
tiendas, entrega de pedidos institucionales). El empaque protege el
producto durante el almacenamiento y transporte, y cumple funciones de
presentación y marketing.

\subsubsection{Formulación y Consumos
Unitarios}\label{formulaciuxf3n-y-consumos-unitarios}

La Tabla~\ref{tbl-formula-produccion} presenta la fórmula de producción
y los consumos unitarios de materiales directos para la elaboración de
pasteles según los datos proporcionados por DULCE DELEITE S.R.L:

\begin{table}

{\caption{{Fórmula de Producción y Costos de Materiales Directos por
Unidad}{\label{tbl-formula-produccion}}}}

\begin{longtable}[]{@{}
  >{\raggedright\arraybackslash}p{(\linewidth - 8\tabcolsep) * \real{0.0980}}
  >{\centering\arraybackslash}p{(\linewidth - 8\tabcolsep) * \real{0.1765}}
  >{\centering\arraybackslash}p{(\linewidth - 8\tabcolsep) * \real{0.3039}}
  >{\centering\arraybackslash}p{(\linewidth - 8\tabcolsep) * \real{0.1961}}
  >{\centering\arraybackslash}p{(\linewidth - 8\tabcolsep) * \real{0.2255}}@{}}
\toprule\noalign{}
\begin{minipage}[b]{\linewidth}\raggedright
Material
\end{minipage} & \begin{minipage}[b]{\linewidth}\centering
Unidad de Medida
\end{minipage} & \begin{minipage}[b]{\linewidth}\centering
Costo Unitario de Compra (S/)
\end{minipage} & \begin{minipage}[b]{\linewidth}\centering
Consumo por Pastel
\end{minipage} & \begin{minipage}[b]{\linewidth}\centering
Costo por Pastel (S/)
\end{minipage} \\
\midrule\noalign{}
\endhead
\bottomrule\noalign{}
\endlastfoot
Harina & kg & 5.50 & 0.30 kg & 1.65 \\
Huevo & unidad & 0.40 & 4.0 unidades & 1.60 \\
Margarina sin sal & kg & 9.00 & 0.15 kg & 1.35 \\
Leche & litro & 3.50 & 0.4 litros & 1.40 \\
Polvo para hornear & kg & 15.00 & 0.01 kg & 0.15 \\
Esencia de vainilla & litro & 1.00 & 0.015 litros & 0.02 \\
Azúcar & kg & 0.80 & 0.25 kg & 0.20 \\
Chocolate & kg & 10.00 & 0.200 kg & 2.00 \\
\textbf{Total} & & & & \textbf{8.37} \\
\end{longtable}

\noindent \emph{Nota.} Datos~proporcionados por DULCE DELEITE S.R.L para
el período analizado (2017).

\end{table}

Analizando esta formulación, se observa que:

\begin{itemize}
\item
  Los ingredientes de mayor impacto en costo son el chocolate (S/ 2.00
  por unidad, 23.9\% del costo de materiales), la harina (S/ 1.65,
  19.7\%), los huevos (S/ 1.60, 19.1\%) y la leche (S/ 1.40, 16.7\%).
  Estos cuatro ingredientes representan el 79.4\% del costo total de
  materiales directos.
\item
  El consumo de 0.30 kg de harina por pastel, junto con 4 huevos, 0.25
  kg de azúcar y 0.15 kg de margarina, sugiere que se trata de pasteles
  de tamaño individual a mediano (probablemente entre 250 y 350 gramos
  de peso final por unidad).
\item
  La inclusión de 200 gramos de chocolate por pastel es significativa,
  indicando que el chocolate es un componente importante del producto
  (posiblemente una cobertura generosa o un pastel de chocolate
  propiamente).
\item
  Los consumos de polvo de hornear (10 gramos por pastel) y esencia de
  vainilla (15 ml por pastel) son consistentes con fórmulas estándar de
  pastelería para productos de este tamaño.
\end{itemize}

\subsubsection{Control de Calidad}\label{control-de-calidad}

Aunque no se detalla explícitamente en la información proporcionada, el
proceso de producción de alimentos requiere controles de calidad en
múltiples etapas:

\begin{itemize}
\tightlist
\item
  \textbf{Control de materias primas}: verificación de frescura, calidad
  y conformidad de ingredientes recibidos.
\item
  \textbf{Control de proceso}: monitoreo de temperaturas de horneado,
  tiempos de mezclado, consistencia de la masa.
\item
  \textbf{Control de producto terminado}: inspección visual,
  verificación de peso, evaluación sensorial (sabor, textura,
  apariencia).
\end{itemize}

El cumplimiento de estándares de inocuidad alimentaria y buenas
prácticas de manufactura es fundamental, aunque empresas del tamaño de
DULCE DELEITE S.R.L pueden no contar con certificaciones formales como
HACCP o ISO 22000, típicas de empresas más grandes.

\subsubsection{Tiempo de Ciclo de
Producción}\label{tiempo-de-ciclo-de-producciuxf3n}

El tiempo total desde el inicio de la preparación hasta la obtención del
producto empacado y listo para venta puede estimarse en aproximadamente
2 a 3 horas por lote, considerando:

\begin{itemize}
\tightlist
\item
  Preparación y mezclado: 30-45 minutos
\item
  Horneado: 30-45 minutos
\item
  Enfriamiento: 30-60 minutos
\item
  Decorado y empaque: 30-45 minutos
\end{itemize}

Para una orden de 2,000 unidades, asumiendo hornos con capacidad para
40-50 pasteles por tanda, se requerirían aproximadamente 40-50 tandas,
lo que podría completarse en varios días de trabajo continuo con los
cinco operarios asignados a la orden.

\section{Análisis de Costos de
Producción}\label{anuxe1lisis-de-costos-de-producciuxf3n}

Esta sección constituye el núcleo del análisis, presentando el cálculo
detallado de cada elemento del costo de producción para la orden de
2,000 unidades de pasteles fabricada por DULCE DELEITE S.R.L. El
análisis se estructura siguiendo los tres componentes fundamentales del
costo de producción: materiales directos, mano de obra directa y gastos
indirectos de fabricación.

\subsection{Costos de Materiales
Directos}\label{costos-de-materiales-directos}

\subsubsection{Metodología de
Cálculo}\label{metodologuxeda-de-cuxe1lculo}

El costo de materiales directos para la orden se determinó multiplicando
el consumo unitario de cada ingrediente por el costo de adquisición por
unidad de medida, y luego multiplicando el resultado por el número de
unidades a producir (2,000 pasteles). La fórmula general es:

\[
\text{Costo Total de Material}_i = \text{Consumo Unitario}_i \times \text{Costo Unitario}_i \times \text{Unidades a Producir}
\]

Los consumos unitarios fueron determinados a partir de la fórmula de
producción estándar utilizada por la empresa, mientras que los costos
unitarios corresponden a los precios de adquisición de materias primas
vigentes en el período analizado.

\subsubsection{Detalle de Materiales Directos por
Insumo}\label{detalle-de-materiales-directos-por-insumo}

La Tabla~\ref{tbl-materiales-directos-detalle} presenta el cálculo
completo de los costos de materiales directos:

\begin{table}

{\caption{{Detalle de Costos de Materiales Directos para 2,000
Pasteles}{\label{tbl-materiales-directos-detalle}}}}

\begin{longtable}[]{@{}
  >{\raggedright\arraybackslash}p{(\linewidth - 12\tabcolsep) * \real{0.0645}}
  >{\centering\arraybackslash}p{(\linewidth - 12\tabcolsep) * \real{0.1452}}
  >{\centering\arraybackslash}p{(\linewidth - 12\tabcolsep) * \real{0.1774}}
  >{\centering\arraybackslash}p{(\linewidth - 12\tabcolsep) * \real{0.1613}}
  >{\centering\arraybackslash}p{(\linewidth - 12\tabcolsep) * \real{0.1855}}
  >{\centering\arraybackslash}p{(\linewidth - 12\tabcolsep) * \real{0.1210}}
  >{\centering\arraybackslash}p{(\linewidth - 12\tabcolsep) * \real{0.1452}}@{}}
\toprule\noalign{}
\begin{minipage}[b]{\linewidth}\raggedright
Insumo
\end{minipage} & \begin{minipage}[b]{\linewidth}\centering
Unidad de Compra
\end{minipage} & \begin{minipage}[b]{\linewidth}\centering
Costo de Compra (S/)
\end{minipage} & \begin{minipage}[b]{\linewidth}\centering
Consumo por Pastel
\end{minipage} & \begin{minipage}[b]{\linewidth}\centering
Costo por Pastel (S/)
\end{minipage} & \begin{minipage}[b]{\linewidth}\centering
Consumo Total
\end{minipage} & \begin{minipage}[b]{\linewidth}\centering
Costo Total (S/)
\end{minipage} \\
\midrule\noalign{}
\endhead
\bottomrule\noalign{}
\endlastfoot
Harina & kg & 5.50 & 0.30 kg & 1.65 & 600 kg & 3,300.00 \\
Huevo & unidad & 0.40 & 4.0 unidades & 1.60 & 8,000 unidades &
3,200.00 \\
Margarina & kg & 9.00 & 0.15 kg & 1.35 & 300 kg & 2,700.00 \\
Leche & litro & 3.50 & 0.4 litros & 1.40 & 800 litros & 2,800.00 \\
Polvo para hornear & kg & 15.00 & 0.01 kg & 0.15 & 20 kg & 300.00 \\
Esencia vainilla & litro & 1.00 & 0.015 litros & 0.02 & 30 litros &
30.00 \\
Azúcar & kg & 0.80 & 0.25 kg & 0.20 & 500 kg & 400.00 \\
Chocolate & kg & 10.00 & 0.200 kg & 2.00 & 400 kg & 4,000.00 \\
\textbf{TOTAL} & & & & \textbf{8.37} & & \textbf{16,730.00} \\
\end{longtable}

\noindent \emph{Nota.} Elaboración~propia con datos proporcionados por
DULCE DELEITE S.R.L (2017).

\end{table}

\subsubsection{Análisis de la Estructura de
Materiales}\label{anuxe1lisis-de-la-estructura-de-materiales}

El análisis de la composición del costo de materiales revela la
siguiente distribución porcentual:

\begin{itemize}
\tightlist
\item
  \textbf{Chocolate}: S/ 4,000.00 (23.91\%) - Es el insumo de mayor
  costo, reflejando su importancia como ingrediente característico y
  diferenciador del producto.
\item
  \textbf{Harina}: S/ 3,300.00 (19.72\%) - Como ingrediente base
  estructural, representa el segundo costo más significativo.
\item
  \textbf{Huevos}: S/ 3,200.00 (19.13\%) - El alto consumo unitario (4
  huevos por pastel) y su función múltiple (estructura, sabor, color)
  justifican su peso en el costo.
\item
  \textbf{Leche}: S/ 2,800.00 (16.74\%) - Ingrediente líquido principal,
  con consumo relativamente alto (0.4 litros por pastel).
\item
  \textbf{Margarina}: S/ 2,700.00 (16.14\%) - Aunque el consumo unitario
  es moderado (0.15 kg), su precio de compra elevado (S/ 9.00/kg) la
  posiciona como el quinto costo más importante.
\item
  \textbf{Azúcar}: S/ 400.00 (2.39\%) - A pesar de un consumo
  significativo (0.25 kg por pastel), su bajo precio (S/ 0.80/kg) limita
  su impacto en el costo total.
\item
  \textbf{Polvo para hornear}: S/ 300.00 (1.79\%) - Ingrediente de alto
  precio unitario (S/ 15.00/kg) pero consumo mínimo (0.01 kg por
  pastel).
\item
  \textbf{Esencia de vainilla}: S/ 30.00 (0.18\%) - Ingrediente de menor
  impacto económico debido a su consumo mínimo (0.015 litros por pastel)
  y bajo precio.
\end{itemize}

Los cinco primeros insumos (chocolate, harina, huevos, leche y
margarina) concentran el 95.64\% del costo total de materiales directos,
lo que sugiere que los esfuerzos de control de costos y negociación con
proveedores deberían focalizarse prioritariamente en estos ingredientes.

\subsubsection{Consideraciones sobre Precios de
Adquisición}\label{consideraciones-sobre-precios-de-adquisiciuxf3n}

Los costos unitarios de adquisición reflejados corresponden a precios de
mercado local en Ayacucho para el año 2017. Es importante señalar varios
aspectos:

\begin{itemize}
\item
  Los precios de materias primas alimentarias, particularmente productos
  agrícolas como huevos, leche y harina, están sujetos a variabilidad
  estacional y de mercado. Una gestión eficiente de costos requiere
  monitoreo continuo de precios y, cuando sea posible, contratos de
  suministro que estabilicen costos.
\item
  El precio del chocolate (S/ 10.00/kg) y de la margarina (S/ 9.00/kg)
  sugiere que se trata de productos de calidad media, no de gama
  premium. La calidad de estos ingredientes impacta directamente en la
  calidad percibida del producto final y puede justificar precios de
  venta más altos o más bajos.
\item
  No se evidencia en los datos una diferenciación entre precios de
  compra al por mayor versus compras menores. Para una producción de
  2,000 unidades, que requiere volúmenes significativos de materiales
  (600 kg de harina, 8,000 huevos, 400 kg de chocolate), sería esperable
  que la empresa obtuviera descuentos por volumen de compra. Si estos
  descuentos existen, ya están reflejados en los costos unitarios
  proporcionados.
\end{itemize}

\subsubsection{Costos No Visibilizados en Materiales
Directos}\label{costos-no-visibilizados-en-materiales-directos}

El análisis de materiales directos presentado considera únicamente el
costo de adquisición de los ingredientes. Sin embargo, existen costos
adicionales relacionados con materiales que pueden no estar
completamente capturados:

\begin{itemize}
\item
  \textbf{Costos de empaque primario}: moldes de papel o aluminio para
  hornear, envoltorios individuales, cajas o bolsas para presentación
  del producto. Estos materiales, aunque no se incorporan al pastel
  mismo, son necesarios para su producción y comercialización.
\item
  \textbf{Mermas y desperdicios}: en la práctica, los consumos reales
  pueden ser ligeramente superiores a los teóricos debido a pérdidas en
  el manejo (derrames, roturas de huevos, residuos que quedan en
  recipientes de mezcla). Si las mermas no están consideradas en la
  formulación de consumos unitarios, el costo real de materiales sería
  superior al calculado.
\item
  \textbf{Costos de almacenamiento}: aunque generalmente clasificados
  como costos indirectos, el mantenimiento de inventarios de materias
  primas (refrigeración para huevos y margarina, almacenamiento en
  condiciones adecuadas para harina y azúcar) implica costos que deben
  cubrirse.
\end{itemize}

La ausencia de estos rubros en el análisis sugiere que el sistema de
costeo de DULCE DELEITE S.R.L se enfoca en los costos directos más
evidentes, pero podría beneficiarse de una identificación más exhaustiva
de costos relacionados con materiales.

\subsection{Costos de Mano de Obra
Directa}\label{costos-de-mano-de-obra-directa}

\subsubsection{Estructura de la Nómina de
Producción}\label{estructura-de-la-nuxf3mina-de-producciuxf3n}

Para la orden analizada, DULCE DELEITE S.R.L asignó cinco trabajadores
de producción directa. El costo total de mano de obra directa reportado
por la empresa asciende a S/ 8,192.63.

Este monto incluye:

\begin{itemize}
\tightlist
\item
  Remuneraciones básicas de los cinco trabajadores
\item
  Cargas sociales obligatorias (EsSalud, SCTR, SENATI)
\item
  Beneficios laborales prorrateados (gratificaciones, CTS, vacaciones)
\end{itemize}

El esquema de asignación de producción revela que cada trabajador fue
responsable de la elaboración de 400 pasteles (equivalente al 20\% de la
orden total), con un costo de mano de obra directa de aproximadamente S/
1,638.53 por trabajador.

\subsubsection{Análisis de Productividad y
Eficiencia}\label{anuxe1lisis-de-productividad-y-eficiencia}

\textbf{Productividad por trabajador:}

\[
\text{Productividad} = \frac{400 \text{ pasteles}}{\text{período de trabajo}}
\]

Asumiendo que la producción se realizó en varios días (dado el tiempo
requerido para horneado, enfriamiento, etc.), cada trabajador debió
producir entre 40 y 80 pasteles por día, dependiendo de la cantidad de
días dedicados a la orden. Esta productividad es razonable considerando
que cada pastel requiere preparación individual de masa, llenado de
moldes, horneado, enfriamiento y acabado.

\textbf{Costo de mano de obra directa por pastel:}

\[
\text{Costo MOD unitario} = \frac{\text{S/ 8,192.63}}{2,000 \text{ pasteles}} = \text{S/ 4.10 por pastel}
\]

Este costo unitario de S/ 4.10 representa aproximadamente el 31.8\% del
costo total unitario de S/ 12.87, lo que indica que la producción de
pasteles es un proceso relativamente intensivo en mano de obra,
característica típica de la industria de panadería y pastelería
artesanal o semi-industrial.

\subsubsection{Consideraciones sobre Estructura
Salarial}\label{consideraciones-sobre-estructura-salarial}

El costo total de mano de obra directa de S/ 8,192.63 para cinco
trabajadores que produjeron 400 pasteles cada uno (2,000 pasteles en
total) incluye todos los componentes del costo laboral en el Perú:

\begin{itemize}
\tightlist
\item
  Salarios básicos
\item
  Cargas sociales (EsSalud 9\%, SCTR 1.04\%, SENATI 0.75\%)
\item
  Beneficios laborales prorrateados (gratificaciones 16.67\% mensual,
  CTS 8.33\% mensual, vacaciones 8.33\% mensual)
\end{itemize}

La inclusión de estos beneficios prorrateados explica por qué el costo
total de mano de obra es significativamente superior a la simple suma de
los salarios básicos. Esto refleja el costo real de emplear trabajadores
bajo el régimen laboral peruano.

\subsubsection{Mano de Obra Indirecta}\label{mano-de-obra-indirecta}

Adicionalmente al costo de mano de obra directa, la empresa reporta un
costo de mano de obra indirecta de S/ 2,030.14, correspondiente a un
trabajador administrativo. Este costo, aunque relacionado con el soporte
general de las operaciones de producción, no se asigna directamente a la
orden de producción sino que se trata como parte de los gastos generales
de la empresa.

\subsection{Gastos Indirectos de
Fabricación}\label{gastos-indirectos-de-fabricaciuxf3n-1}

\subsubsection{Identificación y Clasificación de
Rubros}\label{identificaciuxf3n-y-clasificaciuxf3n-de-rubros}

Los gastos indirectos de fabricación (GIF) comprenden todos los costos
de producción que no pueden identificarse directamente con la orden
específica pero que son necesarios para el proceso productivo general.
La Tabla~\ref{tbl-gif} presenta el detalle de GIF reportados por DULCE
DELEITE S.R.L:

\begin{table}

{\caption{{Gastos Indirectos de Fabricación}{\label{tbl-gif}}}}

\begin{longtable}[]{@{}lc@{}}
\toprule\noalign{}
Rubro & Monto Mensual (S/) \\
\midrule\noalign{}
\endhead
\bottomrule\noalign{}
\endlastfoot
Arriendo & 350.00 \\
Agua & 80.00 \\
Electricidad & 300.00 \\
Teléfono & 10.00 \\
Depreciación & 50.00 \\
Internet & 30.00 \\
\textbf{Total GIF} & \textbf{820.00} \\
\end{longtable}

\noindent \emph{Nota.} Gastos~mensuales correspondientes al período de
análisis (2017). Datos de DULCE DELEITE S.R.L.

\end{table}

\subsubsection{Análisis de la Estructura de
GIF}\label{anuxe1lisis-de-la-estructura-de-gif}

El total de GIF mensuales asciende a S/ 820.00, representando apenas el
3.2\% del costo total de producción (S/ 820.00 / S/ 25,742.63). Esta
proporción relativamente baja de GIF es característica de empresas de
producción artesanal o semi-artesanal, donde los costos de materiales y
mano de obra directa predominan sobre los costos indirectos.

\textbf{Clasificación por naturaleza:}

\begin{itemize}
\item
  \textbf{Costos fijos}: arriendo (S/ 350.00), internet (S/ 30.00),
  teléfono (S/ 10.00) y depreciación (S/ 50.00) totalizan S/ 440.00,
  representando el 53.7\% de los GIF. Estos costos no varían con el
  nivel de producción en el corto plazo.
\item
  \textbf{Costos variables o semi-variables}: electricidad (S/ 300.00) y
  agua (S/ 80.00) totalizan S/ 380.00, representando el 46.3\% de los
  GIF. Aunque tienen un componente fijo (cargo básico mensual), varían
  con el nivel de uso, que a su vez está relacionado con el volumen de
  producción.
\end{itemize}

\textbf{Componentes principales:}

\begin{itemize}
\item
  \textbf{Arriendo (42.7\%)}: el alquiler de las instalaciones de
  producción es el GIF individual más significativo. El monto de S/
  350.00 mensuales sugiere un local de tamaño pequeño a mediano en
  Ayacucho, consistente con una operación de PYME.
\item
  \textbf{Electricidad (36.6\%)}: el alto consumo eléctrico (S/ 300.00
  mensuales) se explica por el uso intensivo de hornos, batidoras y
  equipos de refrigeración. Este es un costo característico de la
  industria de panadería-pastelería.
\item
  \textbf{Agua (9.8\%)}: necesaria para limpieza de equipos, higiene y
  preparación de algunos ingredientes.
\item
  \textbf{Depreciación (6.1\%)}: el monto modesto (S/ 50.00 mensuales)
  sugiere que la empresa cuenta con maquinaria y equipo de valor
  limitado, o que gran parte del equipo ya está depreciado
  contablemente. Para una empresa con hornos industriales, batidoras y
  otros equipos, una depreciación de solo S/ 50.00 mensuales parece
  baja.
\item
  \textbf{Internet (3.7\%) y Teléfono (1.2\%)}: costos de comunicación
  relativamente bajos, posiblemente porque estas herramientas se usan
  principalmente para coordinación de pedidos y gestión administrativa
  básica.
\end{itemize}

\subsubsection{Rubros No Incluidos en los
GIF}\label{rubros-no-incluidos-en-los-gif}

Es notable la ausencia de varios rubros que típicamente forman parte de
los GIF en empresas manufactureras:

\begin{itemize}
\tightlist
\item
  \textbf{Mantenimiento y reparaciones}: de hornos, batidoras,
  instalaciones.
\item
  \textbf{Materiales indirectos}: lubricantes, productos de limpieza,
  combustibles (si se usan hornos a gas).
\item
  \textbf{Seguros}: de maquinaria, instalaciones, responsabilidad civil.
\item
  \textbf{Impuestos}: predial o arbitrios municipales sobre el local de
  producción.
\item
  \textbf{Supervisión}: si existiera personal de supervisión de
  producción, su salario sería un GIF.
\end{itemize}

La ausencia de estos rubros puede deberse a: (1) que efectivamente no se
incurre en ellos (ej. el mantenimiento se realiza esporádicamente y no
mensualmente), (2) que están incluidos en otros rubros no detallados, o
(3) que el sistema de costeo de la empresa tiene limitaciones en la
captura completa de costos indirectos.

\subsubsection{Asignación de GIF a la Orden de
Producción}\label{asignaciuxf3n-de-gif-a-la-orden-de-producciuxf3n}

Un aspecto crítico del costeo por órdenes es determinar cómo asignar los
GIF (que son costos del período, no directamente rastreables a órdenes
específicas) a cada orden de producción.

El análisis proporcionado indica que los GIF totales de S/ 820.00 se
asignaron completamente a la orden de 2,000 pasteles. Esto implica que:

\begin{itemize}
\tightlist
\item
  Durante el mes en que se produjo esta orden, no se produjeron otras
  órdenes significativas, o
\item
  La empresa utiliza un método simplificado de asignación donde todos
  los GIF del período se cargan a la producción de ese período.
\end{itemize}

\textbf{Cálculo del GIF por unidad:}

\[
\text{GIF unitario} = \frac{\text{S/ 820.00}}{2,000 \text{ pasteles}} = \text{S/ 0.41 por pastel}
\]

Este costo unitario de GIF de S/ 0.41 representa apenas el 3.2\% del
costo total unitario, lo que confirma que la estructura de costos de
esta empresa está dominada por materiales directos (65.0\%) y mano de
obra directa (31.8\%).

Si la empresa tuviera múltiples órdenes en producción simultáneamente,
sería necesario utilizar una base de asignación más elaborada. Las bases
típicas podrían ser:

\begin{itemize}
\tightlist
\item
  \textbf{Horas de mano de obra directa}: si se registrara el tiempo
  dedicado a cada orden.
\item
  \textbf{Costo de mano de obra directa}: usando la proporción del costo
  de MOD de cada orden respecto al total.
\item
  \textbf{Unidades producidas}: si todas las órdenes fueran de productos
  similares.
\item
  \textbf{Horas-máquina}: si se registrara el tiempo de uso de hornos y
  otros equipos por orden.
\end{itemize}

En ausencia de información sobre múltiples órdenes o bases de asignación
específicas, se asume que la totalidad de los GIF del período se asigna
a esta orden.

\subsection{Costo Total de Producción y Determinación del Precio de
Venta}\label{costo-total-de-producciuxf3n-y-determinaciuxf3n-del-precio-de-venta}

\subsubsection{Integración de los Tres Elementos del
Costo}\label{integraciuxf3n-de-los-tres-elementos-del-costo}

La Tabla~\ref{tbl-costo-total} consolida los tres elementos del costo de
producción calculados:

\begin{table}

{\caption{{Costo Total de Producción para 2,000
Pasteles}{\label{tbl-costo-total}}}}

\begin{longtable}[]{@{}
  >{\raggedright\arraybackslash}p{(\linewidth - 8\tabcolsep) * \real{0.2000}}
  >{\centering\arraybackslash}p{(\linewidth - 8\tabcolsep) * \real{0.1800}}
  >{\centering\arraybackslash}p{(\linewidth - 8\tabcolsep) * \real{0.1900}}
  >{\centering\arraybackslash}p{(\linewidth - 8\tabcolsep) * \real{0.2100}}
  >{\centering\arraybackslash}p{(\linewidth - 8\tabcolsep) * \real{0.2200}}@{}}
\toprule\noalign{}
\begin{minipage}[b]{\linewidth}\raggedright
Elemento del Costo
\end{minipage} & \begin{minipage}[b]{\linewidth}\centering
Monto Total (S/)
\end{minipage} & \begin{minipage}[b]{\linewidth}\centering
\% del Costo Total
\end{minipage} & \begin{minipage}[b]{\linewidth}\centering
Costo Unitario (S/)
\end{minipage} & \begin{minipage}[b]{\linewidth}\centering
\% del Costo Unitario
\end{minipage} \\
\midrule\noalign{}
\endhead
\bottomrule\noalign{}
\endlastfoot
Materiales Directos & 16,730.00 & 65.0\% & 8.37 & 65.0\% \\
Mano de Obra Directa & 8,192.63 & 31.8\% & 4.10 & 31.8\% \\
Gastos Indirectos de Fabricación & 820.00 & 3.2\% & 0.41 & 3.2\% \\
\textbf{COSTO TOTAL DE PRODUCCIÓN} & \textbf{25,742.63} &
\textbf{100.0\%} & \textbf{12.87} & \textbf{100.0\%} \\
\end{longtable}

\noindent \emph{Nota.} Elaboración~propia con datos de DULCE DELEITE
S.R.L (2017). El costo unitario se obtiene dividiendo el costo total
entre 2,000 unidades.

\end{table}

\subsubsection{Análisis de la Estructura de
Costos}\label{anuxe1lisis-de-la-estructura-de-costos}

La estructura de costos de DULCE DELEITE S.R.L para la producción de
pasteles muestra una clara predominancia de los costos de materiales
directos (65.0\%), seguidos por la mano de obra directa (31.8\%),
mientras que los gastos indirectos de fabricación tienen un peso
marginal (3.2\%).

Esta estructura es típica de empresas de producción artesanal o
semi-artesanal de alimentos, donde:

\begin{itemize}
\tightlist
\item
  Los ingredientes de calidad son fundamentales para el producto final,
  representando la mayor parte del costo.
\item
  El proceso productivo requiere intervención manual significativa
  (preparación, mezclado, decorado), lo que genera costos importantes de
  mano de obra.
\item
  La inversión en maquinaria y la infraestructura es limitada, por lo
  que los costos indirectos son bajos en comparación con industrias más
  capitalizadas.
\end{itemize}

Comparando con estructuras típicas de otras industrias:

\begin{itemize}
\tightlist
\item
  Una empresa de manufactura altamente automatizada podría tener GIF del
  40-60\% del costo total.
\item
  Una empresa de servicios profesionales tendría mano de obra como
  componente dominante (70-80\%).
\item
  Una empresa extractiva (minería, petróleo) tendría materiales
  relativamente bajos pero GIF muy altos.
\end{itemize}

La estructura de DULCE DELEITE S.R.L es coherente con su naturaleza de
empresa de producción de alimentos elaborados de manera semi-artesanal.

\subsubsection{Determinación del Precio de
Venta}\label{determinaciuxf3n-del-precio-de-venta}

El documento proporcionado indica dos cifras relacionadas con el precio:

\begin{itemize}
\tightlist
\item
  \textbf{Utilidad deseada por unidad}: S/ 7.13
\item
  \textbf{Precio de venta por unidad}: S/ 20.00
\end{itemize}

\textbf{Verificación de la coherencia:}

\[
\begin{aligned}
\text{Costo unitario} &= \text{S/ 12.87} \\
\text{Utilidad deseada} &= \text{S/ 7.13} \\
\text{Precio de venta esperado} &= \text{S/ 12.87 + S/ 7.13} = \text{S/ 20.00} \checkmark
\end{aligned}
\]

\textbf{Margen de utilidad:}

\[
\text{Margen de utilidad} = \frac{\text{S/ 7.13}}{\text{S/ 20.00}} \times 100\% = 35.65\%
\]

\[
\text{Margen sobre costo} = \frac{\text{S/ 7.13}}{\text{S/ 12.87}} \times 100\% = 55.40\%
\]

Este margen de utilidad del 35.65\% sobre el precio de venta (o 55.40\%
sobre el costo) es relativamente saludable y consistente con márgenes
típicos en la industria de panadería y pastelería, que suelen oscilar
entre 30\% y 60\% dependiendo del segmento de mercado, el nivel de
competencia y el tipo de producto.

\textbf{Utilidad total de la orden:}

\[
\text{Utilidad total} = \text{S/ 7.13} \times 2,000 \text{ pasteles} = \text{S/ 14,260.00}
\]

Esta utilidad total de S/ 14,260.00 sobre un costo de producción de S/
25,742.63 representa un retorno significativo sobre el capital invertido
en materiales y mano de obra para esta orden específica.

\subsubsection{Análisis del Punto de
Equilibrio}\label{anuxe1lisis-del-punto-de-equilibrio}

Aunque no se proporcionan datos sobre costos fijos totales de la empresa
(solo los GIF asociados a producción), puede realizarse un análisis
simplificado del punto de equilibrio para esta orden específica.

Si asumimos que los GIF de S/ 820.00 son costos fijos (o
predominantemente fijos) y que los materiales directos y mano de obra
directa son costos variables, el punto de equilibrio sería:

\[
\begin{aligned}
\text{Costos fijos (GIF)} &= \text{S/ 820.00} \\
\text{Costo variable unitario} &= \text{S/ 8.37 + S/ 4.10} = \text{S/ 12.47} \\
\text{Precio de venta unitario} &= \text{S/ 20.00} \\
\text{Margen de contribución unitario} &= \text{S/ 20.00 - S/ 12.47} = \text{S/ 7.53}
\end{aligned}
\]

\[
\text{Punto de equilibrio (unidades)} = \frac{\text{S/ 820.00}}{\text{S/ 7.53}} \approx 109 \text{ pasteles}
\]

Esto significa que con la estructura de costos identificada, la empresa
necesitaría vender aproximadamente 109 pasteles para cubrir los costos
fijos (GIF) asociados al período de producción. Cualquier venta por
encima de este nivel generaría utilidades.

Sin embargo, este análisis es limitado porque:

\begin{itemize}
\tightlist
\item
  No considera los costos fijos totales de la empresa (administrativos,
  financieros, comerciales).
\item
  Asume que la estructura de costos variables es lineal (mismo costo por
  unidad independiente del volumen).
\item
  No considera economías o deseconomías de escala en materiales o mano
  de obra.
\end{itemize}

\subsubsection{Comparación con el
Mercado}\label{comparaciuxf3n-con-el-mercado}

Para evaluar la competitividad del precio de S/ 20.00 por pastel, sería
necesario compararlo con precios de mercado para productos similares en
Ayacucho en 2017. Aunque no se dispone de información específica de
mercado en este estudio, pueden hacerse algunas consideraciones
generales:

\begin{itemize}
\tightlist
\item
  Un precio de S/ 20.00 por un pastel individual de aproximadamente
  300-350 gramos con ingredientes de calidad media (chocolate incluido)
  parece posicionarse en el segmento medio del mercado ayacuchano.
\item
  Pastelerías artesanales de gama alta podrían cobrar S/ 25.00-35.00 por
  productos similares.
\item
  Panaderías tradicionales de barrio o productos de menor calidad
  podrían ofrecer productos comparables a S/ 12.00-18.00.
\item
  Cadenas de supermercados con producción masiva podrían ofrecer
  productos más económicos (S/ 10.00-15.00) pero generalmente de calidad
  diferente.
\end{itemize}

El margen de 35.65\% sobre ventas deja espacio para:

\begin{itemize}
\tightlist
\item
  \textbf{Absorber variaciones en costos de materias primas} sin
  necesidad de ajustar inmediatamente los precios de venta al público.
\item
  \textbf{Cubrir gastos administrativos y de ventas} no incluidos en el
  costo de producción, tales como publicidad, comisiones de vendedores,
  gastos de oficina, salarios de personal no productivo.
\item
  \textbf{Ofrecer descuentos ocasionales o promociones} para estimular
  ventas en períodos de baja demanda o para clientes institucionales con
  compras de gran volumen.
\item
  \textbf{Generar rentabilidad neta después de impuestos} suficiente
  para reinversión en equipamiento, expansión de capacidad o
  distribución de utilidades a socios.
\end{itemize}

\subsubsection{Evaluación de Sensibilidad del
Precio}\label{evaluaciuxf3n-de-sensibilidad-del-precio}

Un análisis de sensibilidad permite evaluar cómo variaciones en los
costos afectarían la rentabilidad si el precio de venta se mantiene
constante en S/ 20.00:

\textbf{Escenario 1: Aumento del 10\% en costos de materiales}

\[
\begin{aligned}
\text{Nuevo costo de materiales} &= \text{S/ 16,730.00} \times 1.10 = \text{S/ 18,403.00} \\
\text{Nuevo costo total} &= \text{S/ 18,403.00 + S/ 8,192.63 + S/ 820.00} = \text{S/ 27,415.63} \\
\text{Nuevo costo unitario} &= \text{S/ 27,415.63 / 2,000} = \text{S/ 13.71} \\
\text{Nueva utilidad unitaria} &= \text{S/ 20.00 - S/ 13.71} = \text{S/ 6.29} \\
\text{Nuevo margen} &= \text{S/ 6.29 / S/ 20.00} = 31.45\%
\end{aligned}
\]

El margen se reduciría de 35.65\% a 31.45\%, una caída de 4.2 puntos
porcentuales, pero aún manteniéndose en niveles aceptables.

\textbf{Escenario 2: Aumento del 10\% en costos de mano de obra}

\[
\begin{aligned}
\text{Nuevo costo de MOD} &= \text{S/ 8,192.63} \times 1.10 = \text{S/ 9,011.89} \\
\text{Nuevo costo total} &= \text{S/ 16,730.00 + S/ 9,011.89 + S/ 820.00} = \text{S/ 26,561.89} \\
\text{Nuevo costo unitario} &= \text{S/ 26,561.89 / 2,000} = \text{S/ 13.28} \\
\text{Nueva utilidad unitaria} &= \text{S/ 20.00 - S/ 13.28} = \text{S/ 6.72} \\
\text{Nuevo margen} &= \text{S/ 6.72 / S/ 20.00} = 33.60\%
\end{aligned}
\]

El impacto sería menor que en el caso de materiales, reduciéndose el
margen a 33.60\%, dado que la mano de obra representa una proporción
menor del costo total.

\textbf{Escenario 3: Reducción del precio de venta en 10\%}

\[
\begin{aligned}
\text{Nuevo precio de venta} &= \text{S/ 20.00} \times 0.90 = \text{S/ 18.00} \\
\text{Nueva utilidad unitaria} &= \text{S/ 18.00 - S/ 12.87} = \text{S/ 5.13} \\
\text{Nuevo margen} &= \text{S/ 5.13 / S/ 18.00} = 28.50\%
\end{aligned}
\]

Una reducción del 10\% en el precio de venta comprimiría
significativamente el margen a 28.50\%, lo que podría comprometer la
capacidad de cubrir todos los gastos operativos no incluidos en el costo
de producción.

Estos análisis de sensibilidad demuestran que la empresa tiene cierta
holgura para absorber incrementos moderados en costos, pero que
reducciones en precios tendrían un impacto más severo en la
rentabilidad.

\section{Discusión}\label{discusiuxf3n}

La sección analiza críticamente los hallazgos del estudio de costeo,
interpretando la estructura de costos identificada, evaluando sus
implicaciones para la gestión empresarial y comparando los resultados
con la literatura especializada y parámetros de la industria.

\subsection{Análisis de la Estructura de
Costos}\label{anuxe1lisis-de-la-estructura-de-costos-1}

La estructura de costos identificada en DULCE DELEITE S.R.L (65\%
materiales directos, 31.8\% mano de obra directa, 3.2\% GIF) presenta
características distintivas que merecen un análisis detallado desde
múltiples perspectivas.

\subsubsection{Predominancia de Materiales
Directos}\label{predominancia-de-materiales-directos}

El alto peso de los materiales directos (65\% del costo total) es una
característica típica de la industria de alimentos elaborados, pero al
mismo tiempo representa una vulnerabilidad significativa para la
empresa. Esta predominancia implica que la rentabilidad de DULCE DELEITE
S.R.L es altamente sensible a variaciones en los precios de los insumos,
particularmente de los cinco ingredientes principales que concentran el
95.64\% del costo de materiales.

Como se demostró en el análisis de sensibilidad, un incremento del 10\%
en el costo de materiales reduce el margen de utilidad de 35.65\% a
31.45\%, lo que representa una pérdida de 4.2 puntos porcentuales de
rentabilidad. Esta vulnerabilidad es particularmente relevante
considerando la volatilidad histórica de precios de productos agrícolas
como huevos y leche, que pueden experimentar fluctuaciones estacionales
significativas debido a factores climáticos, variaciones en la oferta y
cambios en la demanda.

La teoría de gestión de costos sugiere varias estrategias para mitigar
esta vulnerabilidad. Las empresas con alta dependencia de materias
primas pueden implementar contratos de suministro a plazo fijo para
estabilizar costos, especialmente para insumos críticos. En el caso de
DULCE DELEITE S.R.L, sería particularmente beneficioso negociar acuerdos
de suministro trimestral o semestral para chocolate, harina y huevos,
que juntos representan el 62.74\% del costo de materiales.

Adicionalmente, empresas con estructuras de costos dominadas por
materiales variables deben monitorear continuamente los mercados de
insumos y estar preparadas para ajustar precios de venta de manera
oportuna cuando los incrementos en costos de materiales excedan umbrales
predeterminados. Para DULCE DELEITE S.R.L, esto implicaría establecer un
sistema de alerta que dispare revisiones de precios cuando el costo de
materiales por pastel exceda, por ejemplo, S/ 9.00 (un 7.5\% por encima
del costo actual de S/ 8.37).

\subsubsection{Intensidad en Mano de Obra y sus
Implicaciones}\label{intensidad-en-mano-de-obra-y-sus-implicaciones}

El componente de mano de obra directa (31.8\% del costo total) refleja
un proceso productivo semi-artesanal donde la intervención humana es
fundamental en múltiples etapas: pesado de ingredientes, mezclado,
llenado de moldes, control del horneado, decorado y empaque. Esta
característica sitúa a DULCE DELEITE S.R.L en un punto intermedio entre
la producción completamente artesanal (que tendría costos de mano de
obra aún mayores) y la producción industrializada (que habría sustituido
parte del trabajo humano con maquinaria automatizada).

Esta estructura tiene ventajas y desventajas claramente identificables:

\textbf{Ventajas del proceso semi-artesanal:}

\begin{enumerate}
\def\labelenumi{\arabic{enumi}.}
\item
  \textbf{Flexibilidad productiva}: la empresa puede adaptar fácilmente
  sus productos a especificaciones de clientes sin necesidad de cambios
  costosos en maquinaria o procesos. Esta flexibilidad es valiosa para
  atender pedidos personalizados o para experimentar con nuevos
  productos.
\item
  \textbf{Diferenciación de calidad}: la intervención artesanal permite
  un control más fino de la calidad en cada etapa, lo que puede
  traducirse en productos de mayor calidad percibida y justificar
  precios superiores a los de productos industriales masivos.
\item
  \textbf{Baja inversión de capital}: la empresa no requiere inversiones
  masivas en equipamiento automatizado, lo que reduce las barreras de
  entrada y los costos fijos de capital.
\end{enumerate}

\textbf{Desafíos del proceso semi-artesanal:}

\begin{enumerate}
\def\labelenumi{\arabic{enumi}.}
\item
  \textbf{Vulnerabilidad a aumentos salariales}: con 31.8\% del costo en
  mano de obra, la empresa es sensible a incrementos del salario mínimo
  o de cargas sociales. El análisis de sensibilidad mostró que un
  aumento del 10\% en costos de mano de obra reduciría el margen a
  33.60\%.
\item
  \textbf{Limitaciones de escalabilidad}: aumentar la producción
  requiere contratar proporcionalmente más trabajadores, lo que limita
  las economías de escala. Una empresa altamente automatizada, en
  contraste, podría duplicar producción con incrementos marginales en
  costos de mano de obra.
\item
  \textbf{Dependencia de disponibilidad de trabajadores capacitados}: la
  calidad del producto depende críticamente de las habilidades de los
  trabajadores. La rotación de personal puede afectar negativamente la
  consistencia del producto.
\end{enumerate}

Empresas intensivas en mano de obra directa deben evaluar cuidadosamente
las decisiones de inversión en automatización mediante análisis de
costo-beneficio que consideren no solo los costos directos sino también
los efectos en calidad, flexibilidad y capacidad de respuesta al
mercado. Para DULCE DELEITE S.R.L, inversiones en equipamiento como
batidoras industriales de mayor capacidad o sistemas de horneado con
control automático de temperatura podrían reducir el tiempo de trabajo
manual y mejorar la consistencia, pero deben evaluarse considerando el
impacto en los GIF (depreciación, mantenimiento, energía) y la potencial
pérdida del carácter artesanal diferenciador del producto.

\subsubsection{Gastos Indirectos de Fabricación Atípicamente
Bajos}\label{gastos-indirectos-de-fabricaciuxf3n-atuxedpicamente-bajos}

Los GIF de solo 3.2\% del costo total son atípicamente bajos, incluso
para empresas de producción artesanal o semi-artesanal. La literatura
reporta que en la industria de panadería-pastelería artesanal, los GIF
típicamente representan entre 5\% y 15\% del costo total, lo que sugiere
que DULCE DELEITE S.R.L puede estar subregistrando costos indirectos.

Varios factores pueden explicar esta situación:

\begin{enumerate}
\def\labelenumi{\arabic{enumi}.}
\item
  \textbf{Subregistro de costos}: como se identificó en el análisis,
  rubros típicos de GIF como mantenimiento, seguros, materiales de
  limpieza y herramientas menores no aparecen en los datos
  proporcionados. Es posible que estos costos se incurran pero no se
  registren sistemáticamente en el sistema contable.
\item
  \textbf{Equipamiento mayormente depreciado}: la depreciación mensual
  de solo S/ 50.00 sugiere que los activos fijos tienen bajo valor
  contable, ya sea porque fueron adquiridos hace mucho tiempo o porque
  la empresa opera con equipamiento de bajo costo. Esto mantiene la
  depreciación baja pero podría implicar riesgos de obsolescencia o
  ineficiencia.
\item
  \textbf{Operación básica sin inversión en infraestructura}: el
  arriendo de S/ 350.00 mensuales y los bajos gastos en comunicaciones
  sugieren que la empresa opera con infraestructura mínima, lo que
  limita los costos fijos pero también puede limitar la capacidad de
  crecimiento.
\end{enumerate}

Las implicaciones de esta estructura de GIF son mixtas:

\textbf{Aspecto positivo}: los bajos costos fijos dan a la empresa una
estructura de costos predominantemente variable (96.8\% del costo total
es variable), lo que implica un punto de equilibrio bajo y menor riesgo
operativo. En períodos de baja demanda, la empresa puede reducir
producción sin cargar con costos fijos elevados.

\textbf{Aspecto negativo}: la baja inversión en infraestructura,
equipamiento y sistemas puede limitar la eficiencia operativa, la
capacidad de escala y la competitividad a largo plazo. La ausencia de
seguros de responsabilidad civil o de activos expone a la empresa a
riesgos financieros significativos en caso de siniestros.

Desde la perspectiva de la gestión estratégica de costos, las empresas
deben equilibrar la minimización de costos con la inversión en
capacidades que generen valor a largo plazo. Para DULCE DELEITE S.R.L,
esto podría implicar inversiones selectivas en áreas como:

\begin{itemize}
\tightlist
\item
  Control de calidad e inocuidad alimentaria (análisis de laboratorio,
  certificaciones)
\item
  Mantenimiento preventivo de equipos para evitar fallas costosas
\item
  Seguros de responsabilidad civil y de activos para mitigar riesgos
\item
  Sistemas de información para gestión de inventarios y costos
\end{itemize}

\subsection{Evaluación del Precio de Venta y Estrategia de
Pricing}\label{evaluaciuxf3n-del-precio-de-venta-y-estrategia-de-pricing}

El precio de S/ 20.00 por pastel, que genera un margen de 35.65\% sobre
ventas, merece una evaluación desde múltiples perspectivas:
competitividad, estrategia de posicionamiento y sostenibilidad.

\subsubsection{Adecuación del Margen}\label{adecuaciuxf3n-del-margen}

El margen de 35.65\% sobre el precio de venta es consistente con los
rangos típicos reportados en la literatura para la industria de
panadería-pastelería. Los márgenes brutos (utilidad bruta / ventas) en
la industria alimentaria varían entre 25\% y 50\%, dependiendo del
segmento de mercado, ubicación geográfica y nivel de diferenciación del
producto.

Sin embargo, es crucial reconocer que este margen bruto de 35.65\% debe
cubrir no solo la utilidad neta sino también gastos operativos no
incluidos en el costo de producción:

\begin{itemize}
\tightlist
\item
  \textbf{Gastos de ventas}: publicidad, promociones, comisiones de
  vendedores, gastos de distribución y entrega
\item
  \textbf{Gastos administrativos}: salarios de gerencia, contabilidad,
  gastos de oficina
\item
  \textbf{Gastos financieros}: intereses sobre préstamos, comisiones
  bancarias
\item
  \textbf{Impuestos}: Impuesto General a las Ventas (IGV) e Impuesto a
  la Renta
\end{itemize}

Si asumimos conservadoramente que estos gastos adicionales representan
el 15-20\% del precio de venta (S/ 3.00 - S/ 4.00 por pastel), el margen
neto real estaría en el rango de 15.65\% a 20.65\%, lo que sigue siendo
aceptable pero deja menos holgura de lo que el margen bruto sugiere.

\subsubsection{Estrategia de Precio Basada en Costos versus Precio de
Mercado}\label{estrategia-de-precio-basada-en-costos-versus-precio-de-mercado}

El análisis revela que DULCE DELEITE S.R.L utiliza una estrategia de
fijación de precios basada en costos (cost-plus pricing), donde el
precio de venta se determina agregando un margen de utilidad deseado al
costo de producción. Esta estrategia es común en empresas pequeñas y
medianas debido a su simplicidad, pero tiene limitaciones importantes:

\begin{enumerate}
\def\labelenumi{\arabic{enumi}.}
\item
  \textbf{Ignora la disposición a pagar de los clientes}: el precio
  óptimo desde la perspectiva del cliente puede ser mayor o menor que
  costo más margen, dependiendo del valor percibido del producto.
\item
  \textbf{No considera precios de competidores}: si los competidores
  ofrecen productos similares a precios significativamente menores, un
  precio basado solo en costos puede resultar en pérdida de
  participación de mercado.
\item
  \textbf{Puede perpetuar ineficiencias}: si los costos son altos debido
  a ineficiencias operativas, una estrategia de cost-plus simplemente
  transfiere esas ineficiencias al cliente vía precios más altos, lo que
  puede no ser sostenible.
\end{enumerate}

Una estrategia más robusta combinaría el análisis de costos con
investigación de mercado sobre precios de competidores y valor percibido
por clientes. Para DULCE DELEITE S.R.L, esto implicaría:

\begin{itemize}
\tightlist
\item
  Realizar un estudio de precios de pasteles similares en Ayacucho
  (pastelerías locales, panaderías, supermercados)
\item
  Evaluar mediante encuestas o pruebas de concepto la disposición a
  pagar de clientes objetivo
\item
  Considerar estrategias de precio diferenciado por canal (precio
  mayorista para distribuidores vs.~precio minorista para clientes
  finales)
\end{itemize}

\subsubsection{Posicionamiento en el
Mercado}\label{posicionamiento-en-el-mercado}

El precio de S/ 20.00 por un pastel de aproximadamente 300-350 gramos
con inclusión significativa de chocolate (200 gramos) posiciona al
producto en el segmento medio del mercado ayacuchano. Esta posición
tiene implicaciones estratégicas:

\textbf{Ventajas del posicionamiento medio:}

\begin{itemize}
\tightlist
\item
  Mercado objetivo más amplio que el segmento premium
\item
  Menor vulnerabilidad a competencia de productos de bajo precio
\item
  Flexibilidad para ajustar precios hacia arriba o hacia abajo según
  condiciones de mercado
\end{itemize}

\textbf{Riesgos del posicionamiento medio:}

\begin{itemize}
\tightlist
\item
  Competencia desde ambos extremos: productos premium que justifican
  precios más altos con calidad superior, y productos económicos que
  compiten por precio
\item
  Dificultad para diferenciarse sin un claro posicionamiento de valor
\end{itemize}

Para fortalecer su posicionamiento, DULCE DELEITE S.R.L podría
considerar estrategias de diferenciación como:

\begin{itemize}
\tightlist
\item
  Certificaciones de calidad o inocuidad alimentaria (ej. HACCP, Digesa)
\item
  Énfasis en insumos de origen local o natural
\item
  Personalización de productos para eventos especiales
\item
  Desarrollo de marca con historia y valores reconocibles
\end{itemize}

\subsection{Limitaciones del Sistema de Costeo
Implementado}\label{limitaciones-del-sistema-de-costeo-implementado}

El análisis realizado ha identificado varias limitaciones significativas
en el sistema de costeo por órdenes implementado por DULCE DELEITE
S.R.L:

\subsubsection{Ausencia de Costeo Estándar y Control de
Variaciones}\label{ausencia-de-costeo-estuxe1ndar-y-control-de-variaciones}

La empresa trabaja exclusivamente con costos históricos o reales, sin
utilizar costos estándar o presupuestados como herramienta de control.
Los sistemas de costeo estándar permiten:

\begin{enumerate}
\def\labelenumi{\arabic{enumi}.}
\item
  \textbf{Establecer objetivos de costos}: los estándares sirven como
  metas de eficiencia para trabajadores y gestores
\item
  \textbf{Identificar variaciones}: la comparación entre costos reales y
  estándares permite detectar ineficiencias, desperdicios o cambios en
  precios de insumos
\item
  \textbf{Facilitar la toma de decisiones}: los estándares proporcionan
  una base más estable para cotizar nuevos pedidos que los costos
  históricos, que pueden estar afectados por ineficiencias temporales
\end{enumerate}

Para DULCE DELEITE S.R.L, implementar un sistema básico de costeo
estándar implicaría:

\begin{itemize}
\tightlist
\item
  Establecer consumos estándar de materiales por pastel basados en la
  fórmula de producción (ya existente)
\item
  Definir tiempos estándar de mano de obra por unidad o por lote
\item
  Calcular tasas estándar de GIF
\item
  Comparar mensualmente costos reales vs.~estándares e investigar
  variaciones significativas
\end{itemize}

\subsubsection{Asignación Simplificada de Gastos
Indirectos}\label{asignaciuxf3n-simplificada-de-gastos-indirectos}

El método actual de asignar la totalidad de los GIF mensuales a la
producción del período es excesivamente simplista. Este enfoque solo es
apropiado si:

\begin{enumerate}
\def\labelenumi{\arabic{enumi}.}
\tightlist
\item
  La empresa produce un solo tipo de producto
\item
  No hay producción en proceso al inicio o fin del período
\item
  No hay inventarios significativos de productos terminados
\end{enumerate}

Si DULCE DELEITE S.R.L produce múltiples tipos de pasteles o tiene otras
líneas de productos (lo cual es probable en una pastelería), este método
podría estar distorsionando los costos de productos individuales.

Un método más robusto requeriría:

\begin{itemize}
\tightlist
\item
  Calcular una tasa predeterminada de GIF al inicio del período
\item
  Seleccionar una base de asignación apropiada (horas de MOD, costo de
  MOD, unidades producidas)
\item
  Aplicar GIF a cada orden según su consumo de la base de asignación
\item
  Conciliar al final del período los GIF aplicados con los GIF reales
  incurridos, reconociendo sobre/subaplicación
\end{itemize}

\subsubsection{Falta de Diferenciación entre Costos Fijos y
Variables}\label{falta-de-diferenciaciuxf3n-entre-costos-fijos-y-variables}

El sistema no distingue explícitamente entre costos fijos y variables,
lo que limita significativamente el análisis gerencial. Esta distinción
es fundamental para:

\begin{enumerate}
\def\labelenumi{\arabic{enumi}.}
\item
  \textbf{Análisis de contribución marginal}: entender cuánto contribuye
  cada unidad vendida a cubrir costos fijos y generar utilidad
\item
  \textbf{Decisiones sobre pedidos especiales}: evaluar si aceptar
  pedidos a precios reducidos que cubran costos variables y contribuyan
  parcialmente a costos fijos
\item
  \textbf{Análisis de punto de equilibrio preciso}: determinar el
  volumen de ventas necesario para cubrir todos los costos
\item
  \textbf{Planificación de utilidades}: modelar cómo diferentes
  escenarios de volumen afectan la rentabilidad
\end{enumerate}

Para DULCE DELEITE S.R.L, una reclasificación de costos sería
relativamente sencilla:

\textbf{Costos variables:}

\begin{itemize}
\tightlist
\item
  Materiales directos: S/ 16,730.00 (100\% variable)
\item
  Mano de obra directa: S/ 8,192.63 (variable si el pago es por
  producción; si es salario fijo, sería costo fijo)
\item
  GIF variables: Electricidad y agua (estimado S/ 380.00, aunque con
  componente fijo)
\end{itemize}

\textbf{Costos fijos:}

\begin{itemize}
\tightlist
\item
  GIF fijos: Arriendo, internet, teléfono, depreciación (S/ 440.00)
\end{itemize}

Esta reclasificación permitiría calcular un margen de contribución más
preciso y realizar análisis más sofisticados de rentabilidad.

\subsubsection{Probable Subregistro de Costos
Indirectos}\label{probable-subregistro-de-costos-indirectos}

Como se discutió anteriormente, la ausencia de rubros típicos de GIF
como mantenimiento, seguros, materiales indirectos y herramientas
sugiere que el costo real de producción puede ser superior al calculado.
Este subregistro puede generar decisiones erróneas de fijación de
precios o aceptación de pedidos.

Se recomienda que las empresas implementen sistemas de captura
exhaustiva de costos, incluyendo:

\begin{itemize}
\tightlist
\item
  Registros de mantenimiento preventivo y correctivo
\item
  Control de consumo de materiales indirectos (productos de limpieza,
  lubricantes, herramientas menores)
\item
  Registro de tiempos de personal indirecto (supervisión, control de
  calidad)
\item
  Inventarios de repuestos y materiales auxiliares
\end{itemize}

\subsection{Comparación con Literatura y Estándares de la
Industria}\label{comparaciuxf3n-con-literatura-y-estuxe1ndares-de-la-industria}

Los resultados de este estudio pueden compararse con hallazgos de
investigaciones similares sobre costeo en la industria de panadería y
pastelería.

Estudios sobre pequeñas panaderías artesanales en México encontraron
estructuras de costos con rangos de:

\begin{itemize}
\tightlist
\item
  Materiales: 55-70\% del costo total
\item
  Mano de obra: 25-35\%
\item
  GIF: 5-15\%
\end{itemize}

DULCE DELEITE S.R.L se ubica dentro de estos rangos en materiales (65\%)
y mano de obra (31.8\%), pero en el límite inferior de GIF (3.2\%),
reforzando la hipótesis de subregistro de costos indirectos.

En cuanto a márgenes de utilidad, la literatura reporta que pastelerías
artesanales en América Latina generalmente operan con márgenes brutos
entre 30\% y 50\%. El margen de 35.65\% de DULCE DELEITE S.R.L se sitúa
cómodamente dentro de este rango, sugiriendo que la empresa tiene un
precio competitivo sin estar en los extremos de precio bajo o premium.

Un aspecto interesante es la comparación de productividad de mano de
obra. Con 5 trabajadores produciendo 2,000 pasteles, la productividad es
de 400 pasteles por trabajador. Sin datos de tiempo exacto de
producción, es difícil calcular pasteles por hora-hombre, pero asumiendo
una semana de producción (40 horas por trabajador), la productividad
sería de 10 pasteles por hora-hombre. Esta cifra es consistente con
procesos semi-artesanales que requieren preparación individualizada de
cada pastel.

\section{Conclusiones}\label{conclusiones}

Con base en el análisis exhaustivo del sistema de costeo por órdenes de
producción implementado en la empresa DULCE DELEITE S.R.L, se establecen
las siguientes conclusiones:

\begin{enumerate}
\def\labelenumi{\arabic{enumi}.}
\item
  \textbf{Estructura de Costos Identificada}: El costo total de
  producción para una orden de 2,000 pasteles asciende a S/ 25,742.63,
  equivalente a un costo unitario de S/ 12.87. La estructura se
  caracteriza por una marcada predominancia de materiales directos
  (65.0\% del costo total), seguida por mano de obra directa (31.8\%) y
  gastos indirectos de fabricación marginales (3.2\%). Esta estructura
  es típica de empresas de producción semi-artesanal de alimentos, donde
  los ingredientes de calidad y el trabajo manual son fundamentales para
  el producto final.
\item
  \textbf{Costos de Materiales Directos y su Concentración}: Los
  materiales directos totalizan S/ 16,730.00 para la orden completa. Los
  cinco insumos principales (chocolate, harina, huevos, leche y
  margarina) concentran el 95.64\% del costo de materiales, lo que
  sugiere que los esfuerzos de control de costos, negociación con
  proveedores y gestión de inventarios deberían focalizarse
  prioritariamente en estos ingredientes críticos.
\item
  \textbf{Costos de Mano de Obra y Cargas Sociales}: El costo de mano de
  obra directa de S/ 8,192.63 incluye remuneraciones básicas de cinco
  trabajadores más todas las cargas sociales y beneficios laborales
  exigidos por la legislación peruana vigente (EsSalud, SCTR, SENATI,
  gratificaciones prorrateadas, CTS y vacaciones). El costo unitario de
  mano de obra de S/ 4.10 por pastel (31.8\% del costo total) refleja un
  proceso productivo intensivo en trabajo manual, característico de la
  pastelería artesanal o semi-artesanal.
\item
  \textbf{Gastos Indirectos Atípicamente Bajos}: Los GIF de S/ 820.00
  mensuales (S/ 0.41 por pastel, 3.2\% del costo total) son atípicamente
  bajos en comparación con estándares de la industria (5-15\%). Esto
  sugiere infraestructura básica, bajo nivel de inversión en
  equipamiento y posible subregistro de costos indirectos como
  mantenimiento, seguros, materiales de limpieza y herramientas menores.
  Esta situación, si bien mantiene costos fijos bajos, puede limitar la
  capacidad de crecimiento y exponer a la empresa a riesgos operativos.
\item
  \textbf{Rentabilidad y Precio de Venta}: Con un precio de venta de S/
  20.00 por pastel, la empresa obtiene un margen de utilidad bruto de
  35.65\% sobre el precio de venta (55.40\% sobre el costo), generando
  una utilidad total de S/ 14,260.00 para la orden de 2,000 unidades.
  Este margen es saludable y consistente con rangos típicos de la
  industria (30-50\%), pero debe cubrir gastos administrativos, de
  ventas, financieros e impuestos no incluidos en el costo de
  producción.
\item
  \textbf{Viabilidad del Sistema de Costeo por Órdenes}: El sistema de
  costeo por órdenes es apropiado y funcional para DULCE DELEITE S.R.L
  dado su esquema de producción por pedidos o lotes específicos. Permite
  identificar costos por orden, calcular rentabilidad por producto y
  proporcionar información para decisiones de fijación de precios. Sin
  embargo, el sistema actual es básico y podría beneficiarse
  significativamente de mejoras en la captura de costos indirectos,
  diferenciación entre costos fijos y variables, e implementación de
  costeo estándar.
\item
  \textbf{Vulnerabilidad a Variaciones de Costos}: La alta proporción de
  materiales directos (65\%) hace que la rentabilidad de la empresa sea
  altamente sensible a variaciones en precios de insumos. El análisis de
  sensibilidad demostró que un incremento del 10\% en costos de
  materiales reduciría el margen de utilidad de 35.65\% a 31.45\%. Esta
  vulnerabilidad requiere estrategias proactivas de gestión de riesgos
  como contratos de suministro a plazo fijo, monitoreo de mercados de
  insumos y mecanismos ágiles de ajuste de precios.
\item
  \textbf{Punto de Equilibrio Favorable}: Considerando solo los GIF como
  costos fijos (S/ 820.00) y clasificando materiales y mano de obra como
  costos variables, el punto de equilibrio se ubica en aproximadamente
  109 pasteles, muy por debajo de la producción de la orden analizada
  (2,000 unidades). Esto indica una estructura de costos
  predominantemente variable (96.8\%), lo que reduce el riesgo operativo
  en períodos de baja demanda pero también limita las economías de
  escala.
\item
  \textbf{Limitaciones Identificadas del Sistema}: El análisis reveló
  varias limitaciones significativas: (1) ausencia de costeo estándar
  para control de eficiencia y variaciones, (2) asignación simplificada
  de GIF que no diferencia entre órdenes cuando hay producción múltiple,
  (3) falta de distinción explícita entre costos fijos y variables que
  limita el análisis de contribución marginal, (4) probable subregistro
  de costos indirectos que podría subestimar el costo real de
  producción, y (5) ausencia de sistemas de control de mermas y
  desperdicios.
\item
  \textbf{Contribución al Conocimiento y Aplicabilidad}: Esta
  investigación demuestra empíricamente la viabilidad y utilidad del
  sistema de costeo por órdenes en pequeñas y medianas empresas del
  sector alimentario peruano, específicamente en el rubro de pastelería.
  El estudio proporciona un caso real documentado que puede servir como
  referencia para otras empresas del sector que deseen implementar o
  mejorar sus sistemas de costeo. Las conclusiones resaltan la
  importancia crítica de una correcta determinación de costos para la
  toma de decisiones empresariales informadas en áreas como fijación de
  precios competitivos, control de costos operativos, evaluación de
  rentabilidad por producto y planificación de la producción.
\end{enumerate}

\appendix

\section{Apéndices}\label{apx-apuxe9ndices}

\section{}\label{}

\subsection{Hoja de Costos por Orden de
Producción}\label{apx-hoja-costos}

A continuación se presenta un modelo de hoja de costos por orden que
podría implementarse de manera más formal en DULCE DELEITE S.R.L:

\textbf{HOJA DE COSTOS POR ORDEN DE PRODUCCIÓN}

\begin{longtable}[]{@{}ll@{}}
\toprule\noalign{}
Campo & Información \\
\midrule\noalign{}
\endhead
\bottomrule\noalign{}
\endlastfoot
N° de Orden & 001-2017 \\
Producto & Pasteles con cobertura de chocolate \\
Cantidad & 2,000 unidades \\
Cliente/Destino & {[}Especificar{]} \\
Fecha de inicio & {[}DD/MM/AAAA{]} \\
Fecha de terminación & {[}DD/MM/AAAA{]} \\
Fecha de entrega & {[}DD/MM/AAAA{]} \\
\end{longtable}

\textbf{MATERIALES DIRECTOS}

\begin{longtable}[]{@{}
  >{\raggedright\arraybackslash}p{(\linewidth - 10\tabcolsep) * \real{0.0972}}
  >{\raggedright\arraybackslash}p{(\linewidth - 10\tabcolsep) * \real{0.2222}}
  >{\raggedright\arraybackslash}p{(\linewidth - 10\tabcolsep) * \real{0.1389}}
  >{\centering\arraybackslash}p{(\linewidth - 10\tabcolsep) * \real{0.1389}}
  >{\centering\arraybackslash}p{(\linewidth - 10\tabcolsep) * \real{0.2222}}
  >{\centering\arraybackslash}p{(\linewidth - 10\tabcolsep) * \real{0.1806}}@{}}
\toprule\noalign{}
\begin{minipage}[b]{\linewidth}\raggedright
Fecha
\end{minipage} & \begin{minipage}[b]{\linewidth}\raggedright
Requisición N°
\end{minipage} & \begin{minipage}[b]{\linewidth}\raggedright
Material
\end{minipage} & \begin{minipage}[b]{\linewidth}\centering
Cantidad
\end{minipage} & \begin{minipage}[b]{\linewidth}\centering
Costo Unitario
\end{minipage} & \begin{minipage}[b]{\linewidth}\centering
Costo Total
\end{minipage} \\
\midrule\noalign{}
\endhead
\bottomrule\noalign{}
\endlastfoot
& & Harina & 600 kg & 5.50 & 3,300.00 \\
& & Huevos & 8,000 u & 0.40 & 3,200.00 \\
& & Margarina & 300 kg & 9.00 & 2,700.00 \\
& & Leche & 800 L & 3.50 & 2,800.00 \\
& & Polvo hornear & 20 kg & 15.00 & 300.00 \\
& & Esencia vainilla & 30 L & 1.00 & 30.00 \\
& & Azúcar & 500 kg & 0.80 & 400.00 \\
& & Chocolate & 400 kg & 10.00 & 4,000.00 \\
& & \textbf{TOTAL MATERIALES DIRECTOS} & & & \textbf{16,730.00} \\
\end{longtable}

\textbf{MANO DE OBRA DIRECTA}

\begin{longtable}[]{@{}lccc@{}}
\toprule\noalign{}
Trabajador & Horas & Tasa Horaria & Costo Total \\
\midrule\noalign{}
\endhead
\bottomrule\noalign{}
\endlastfoot
Trabajador 1 & {[}horas{]} & {[}tasa{]} & 1,638.53 \\
Trabajador 2 & {[}horas{]} & {[}tasa{]} & 1,638.53 \\
Trabajador 3 & {[}horas{]} & {[}tasa{]} & 1,638.53 \\
Trabajador 4 & {[}horas{]} & {[}tasa{]} & 1,638.53 \\
Trabajador 5 & {[}horas{]} & {[}tasa{]} & 1,638.53 \\
\textbf{TOTAL MANO DE OBRA DIRECTA} & & & \textbf{8,192.63} \\
\end{longtable}

\textbf{GASTOS INDIRECTOS DE FABRICACIÓN}

\begin{longtable}[]{@{}lcc@{}}
\toprule\noalign{}
Base de Asignación & Tasa & Monto \\
\midrule\noalign{}
\endhead
\bottomrule\noalign{}
\endlastfoot
{[}Base seleccionada{]} & {[}Tasa predeterminada{]} & 820.00 \\
\textbf{TOTAL GIF} & & \textbf{820.00} \\
\end{longtable}

\textbf{RESUMEN DE COSTOS}

\begin{table}

{\caption{{Modelo de Hoja de Costos por
Orden}{\label{tbl-hoja-costos}}}}

\begin{longtable}[]{@{}lc@{}}
\toprule\noalign{}
Elemento & Monto (S/) \\
\midrule\noalign{}
\endhead
\bottomrule\noalign{}
\endlastfoot
Materiales Directos & 16,730.00 \\
Mano de Obra Directa & 8,192.63 \\
Gastos Indirectos de Fabricación & 820.00 \\
\textbf{COSTO TOTAL} & \textbf{25,742.63} \\
\textbf{COSTO UNITARIO} (÷ 2,000) & \textbf{12.87} \\
Utilidad deseada por unidad & 7.13 \\
\textbf{PRECIO DE VENTA UNITARIO} & \textbf{20.00} \\
\end{longtable}

\noindent \emph{Nota.} Modelo~propuesto para implementación formal en
DULCE DELEITE S.R.L.

\end{table}

\section{}\label{}

\subsection{Formulario de Requisición de
Materiales}\label{apx-requisicion}

\textbf{REQUISICIÓN DE MATERIALES}

\begin{longtable}[]{@{}ll@{}}
\toprule\noalign{}
Campo & Información \\
\midrule\noalign{}
\endhead
\bottomrule\noalign{}
\endlastfoot
N° de Requisición & {[}Número correlativo{]} \\
Fecha & {[}DD/MM/AAAA{]} \\
Orden de Producción N° & {[}Número de orden{]} \\
Departamento solicitante & Producción \\
\end{longtable}

\textbf{MATERIALES SOLICITADOS}

\begin{longtable}[]{@{}
  >{\centering\arraybackslash}p{(\linewidth - 10\tabcolsep) * \real{0.0941}}
  >{\raggedright\arraybackslash}p{(\linewidth - 10\tabcolsep) * \real{0.1529}}
  >{\centering\arraybackslash}p{(\linewidth - 10\tabcolsep) * \real{0.0941}}
  >{\centering\arraybackslash}p{(\linewidth - 10\tabcolsep) * \real{0.2471}}
  >{\centering\arraybackslash}p{(\linewidth - 10\tabcolsep) * \real{0.2353}}
  >{\raggedright\arraybackslash}p{(\linewidth - 10\tabcolsep) * \real{0.1765}}@{}}
\toprule\noalign{}
\begin{minipage}[b]{\linewidth}\centering
Código
\end{minipage} & \begin{minipage}[b]{\linewidth}\raggedright
Descripción
\end{minipage} & \begin{minipage}[b]{\linewidth}\centering
Unidad
\end{minipage} & \begin{minipage}[b]{\linewidth}\centering
Cantidad Solicitada
\end{minipage} & \begin{minipage}[b]{\linewidth}\centering
Cantidad Entregada
\end{minipage} & \begin{minipage}[b]{\linewidth}\raggedright
Observaciones
\end{minipage} \\
\midrule\noalign{}
\endhead
\bottomrule\noalign{}
\endlastfoot
& & & & & \\
& & & & & \\
\end{longtable}

\begin{table}

{\caption{{Formulario de Requisición de
Materiales}{\label{tbl-requisicion}}}}

\begin{longtable}[]{@{}llll@{}}
\toprule\noalign{}
Solicitado por & Autorizado por & Entregado por & Recibido por \\
\midrule\noalign{}
\endhead
\bottomrule\noalign{}
\endlastfoot
Firma: & Firma: & Firma: & Firma: \\
Nombre: & Nombre: & Nombre: & Nombre: \\
Fecha: & Fecha: & Fecha: & Fecha: \\
\end{longtable}

\noindent \emph{Nota.} Documento~para control de salida de materiales
del almacén hacia producción.

\end{table}

\section{}\label{}

\subsection{Registro de Tiempo de Mano de Obra}\label{apx-tiempo}

\textbf{TARJETA DE TIEMPO - MANO DE OBRA DIRECTA}

\begin{longtable}[]{@{}ll@{}}
\toprule\noalign{}
Campo & Información \\
\midrule\noalign{}
\endhead
\bottomrule\noalign{}
\endlastfoot
Trabajador & {[}Nombre completo{]} \\
Semana & Del {[}DD/MM/AAAA{]} al {[}DD/MM/AAAA{]} \\
\end{longtable}

\textbf{DISTRIBUCIÓN DE HORAS POR ORDEN}

\begin{longtable}[]{@{}
  >{\raggedright\arraybackslash}p{(\linewidth - 12\tabcolsep) * \real{0.0843}}
  >{\raggedright\arraybackslash}p{(\linewidth - 12\tabcolsep) * \real{0.1205}}
  >{\centering\arraybackslash}p{(\linewidth - 12\tabcolsep) * \real{0.1566}}
  >{\centering\arraybackslash}p{(\linewidth - 12\tabcolsep) * \real{0.1687}}
  >{\centering\arraybackslash}p{(\linewidth - 12\tabcolsep) * \real{0.1566}}
  >{\raggedright\arraybackslash}p{(\linewidth - 12\tabcolsep) * \real{0.1325}}
  >{\raggedright\arraybackslash}p{(\linewidth - 12\tabcolsep) * \real{0.1807}}@{}}
\toprule\noalign{}
\begin{minipage}[b]{\linewidth}\raggedright
Fecha
\end{minipage} & \begin{minipage}[b]{\linewidth}\raggedright
Orden N°
\end{minipage} & \begin{minipage}[b]{\linewidth}\centering
Hora Inicio
\end{minipage} & \begin{minipage}[b]{\linewidth}\centering
Hora Término
\end{minipage} & \begin{minipage}[b]{\linewidth}\centering
Total Horas
\end{minipage} & \begin{minipage}[b]{\linewidth}\raggedright
Actividad
\end{minipage} & \begin{minipage}[b]{\linewidth}\raggedright
Observaciones
\end{minipage} \\
\midrule\noalign{}
\endhead
\bottomrule\noalign{}
\endlastfoot
& & & & & & \\
& & & & & & \\
\textbf{TOTAL HORAS DE LA SEMANA} & & & & & & \\
\end{longtable}

\begin{table}

{\caption{{Tarjeta de Tiempo de Mano de Obra
Directa}{\label{tbl-tiempo}}}}

\begin{longtable}[]{@{}ll@{}}
\toprule\noalign{}
Trabajador & Supervisor \\
\midrule\noalign{}
\endhead
\bottomrule\noalign{}
\endlastfoot
Firma: & Firma: \\
Fecha: & Fecha: \\
\end{longtable}

\noindent \emph{Nota.} Documento~para registro de horas trabajadas por
orden de producción.

\end{table}

\section{Publicaciones Similares}\label{apx-publicaciones-similares}

Si te interesó este artículo, te recomendamos que explores otros blogs y
recursos relacionados que pueden ampliar tus conocimientos. Aquí te dejo
algunas sugerencias:

\begin{enumerate}
\def\labelenumi{\arabic{enumi}.}
\tightlist
\item
  \href{https://actus-mercator.netlify.app/posts/2020-09-15-plan-de-negocio-exportacion-de-trucha-arcoires/index.pdf}{\faIcon{file-pdf}}
  \href{https://actus-mercator.netlify.app/posts/2020-09-15-plan-de-negocio-exportacion-de-trucha-arcoires}{Plan
  De Negocio Exportacion De Trucha Arcoires}
\item
  \href{https://actus-mercator.netlify.app/posts/2021-07-13-plan-de-negocio-exportacion-de-tuna/index.pdf}{\faIcon{file-pdf}}
  \href{https://actus-mercator.netlify.app/posts/2021-07-13-plan-de-negocio-exportacion-de-tuna}{Plan
  De Negocio Exportacion De Tuna}
\item
  \href{https://actus-mercator.netlify.app/posts/2022-01-23-cadena-de-suministros/index.pdf}{\faIcon{file-pdf}}
  \href{https://actus-mercator.netlify.app/posts/2022-01-23-cadena-de-suministros}{Cadena
  De Suministros}
\end{enumerate}

Esperamos que encuentres estas publicaciones igualmente interesantes y
útiles. ¡Disfruta de la lectura!






\end{document}
